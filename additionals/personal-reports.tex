
\section{Persönliche Berichte}
\label{Persönliche Berichte}

\subsection{Robin Suter}
\label{Persönliche Berichte:Robin Suter}

Während ich mit der Studienarbeit im letzten Semester erste Erfahrungen mit Geodaten und insbesondere \gls{OpenStreetMap} machen konnte, bot mir diese Arbeit eine gute Gelegenheit, diese Kenntnisse zu vertiefen und sie auf eine neue Problemdomäne anzuwenden.
Spannend war insbesondere die Verbindung von Geodaten und Verkehrsplanung.
Das Erarbeiten der Spezifikation hat mir einen guten Einblick in dieses Gebiet verschafft.

Aus der technischen Perspektive fand ich vor allem die Arbeit mit PostgreSQL beziehungsweise PostGIS sehr interessant.
Ich konnte ein vertieftes Verständnis von SQL erlangen und lernte viel über die Optimierung von SQL-Queries.

Die Arbeit im Team mit Jonas Matter ist sehr gut gelungen.
Es gab viele Momente, an denen wir mit gemeinsamen Austausch von Ideen rasch zu neuen Lösungsmöglichkeiten gelangten.
Auch unser Prozess der gegenseitigen Reviews hat uns geholfen, die Qualität kontinuierlich hoch zu halten.

Insgesamt bin ich mit dem Verlauf und dem Ausgang dieser Bachelorarbeit sehr zufrieden.
Wir haben alle Ziele erreicht, die wir zu Beginn ausgelegt haben und die Zeitplanung ist zum Ende hin optimal aufgegangen.
Ich hoffe, dass unsere erarbeitete Spezifikation auf Interesse stösst und in Zukunft weiter entwickelt wird.

\subsection{Jonas Matter}
\label{Persönliche Berichte:Jonas Matter}

Die branchenübergreifende Arbeit hat mein Interesse von Anfang an geweckt.
Die Verbindung von grossen Mengen an Daten, dessen Verarbeitung und der Einsatz von PostGIS und pgRouting versprach eine gute Kombination.
Grosse Freude haben mir die verschiedenen Optimierungen, welche der Generator durchlebt hat, bereitet.

Die Zusammenarbeit mit Robin Suter war wie in vorherigen Projekten optimal.
Die regelmässigen Reviews sowie Pair Programming haben zu einer stetigen Steigerung der Qualität geführt.
Durch Hinterfragen der gegenseitigen Lösungen wurde man auf neue Wege geführt, welche man in Einzeilarbeit nicht eingeschlagen hätte.

Abschliessend kann ich sagen, dass ich mit dem Verlauf und dem Resultat der Bachelorarbeit sehr zufrieden bin.
Ich hoffe, dass die ÖV-Güteklassen 2018 bei den unterschiedlichen Stakeholdern Anklang finden wird, sei dies durch die Verwendung der Web-App und des Generators oder der Weiterverwendung und Verbesserung der Spezifikation.
