% Append the following to the XeLaTex command: |makeglossaries %

\newglossaryentry{Geocoding}{
    name={Geocoding},
    description={Der Prozess, einer Postadresse eine Koordinate zuzuordnen. Der umgekehrte Weg, das Bestimmen einer Postadresse aus einer Koordinate, nennt sich \emph{Reverse Geocoding}.}
}

\newglossaryentry{OpenStreetMap}{
    name={OpenStreetMap},
    description={Ein Community-Projekt mit dem Ziel, eine frei verfügbare Karte der Erde zu erstellen, die von jedem bearbeitet und ergänzt werden kann.}
}

\newglossaryentry{ÖV-Güteklassen}{
    name={ÖV-Güteklassen},
    description={ÖV-Güteklassen geben Auskunft darüber, wie gut ein Standort mit dem öffentlichen Verkehr erschlossen ist.}
}

\newglossaryentry{OeVGK93}{
    name={OeVGK93},
    description={Steht kurz für ÖV-Güteklassen 93 und bezeichnet die Definition der ÖV-Güteklassen, welche im Jahre 1993 mit der Schweizer Norm 640 290 verabschiedet wurde.}
}

\newglossaryentry{OeVGK18}{
    name={OeVGK18},
    description={Steht kurz für ÖV-Güteklassen 2018 und bezeichnet die neue Spezifikation, welche im Zuge dieser Arbeit erarbeitet wird.}
}

\newglossaryentry{GeoJSON}{
    name={GeoJSON},
    description={Ein spezifiziertes Format (RFC 7946) für die serialisierte Repräsentation von geografischen Daten. Dafür wird die JSON-Notation verwendet.}
}

\newglossaryentry{Terrainmodell}{
    name={Terrainmodell},
    description={Ein digitales Terrainmodell (DTM) beschreibt die Geländeform ohne Bewuchs und Bebauung.}
}

\newglossaryentry{Leistungskilometer}{
    name={Leistungskilometer},
    description={Der Leistungskilometer berücksichtigt die Horizontaltdistanz sowie die Werte, die sich aus Steigung und starker Gefälle errechnen lassen und ist ein Mass zur Abschätzung des Zeit- und Energieaufwands.}
}

\newglossaryentry{GEOS}{
    name={GEOS},
    description={Steht für "`Geometry Engine, Open Source"', eine Software-Bibliothek, die ein Objektmodell für geometrische Daten bietet und geometrische Funktionen implementiert.}
}

\newglossaryentry{Stored Procedure}{
    name={Stored Procedure},
    plural={Stored Procedures},
    description={Ist eine Funktion in einen Datenbankmanagementsystem, welche aufgerufen werden kann. Dadurch können häufig verwendete Abläufe, die sonst durch den Client ausgeführt werden müssen, auf das Datenbanksystem ausgelagert werden.}
}

\newglossaryentry{Isochrone}{
    name={Isochrone},
    plural={Isochronen},
    description={Eine Darstellung der Erreichbarkeit auf einer Karte von einem Standort aus. Definiert wird dies als geschlossene Linie um einen Standort, die alle Punkte miteinander verbindet, die mit gleicher Laufzeit von diesem Standort entfernt sind.}
}

%TODO GeoJSON Feature