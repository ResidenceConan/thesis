
\section{Analyse}
\label{Analyse}

%TODO

\subsection{Evaluation Frontend-Framework}
\label{Analyse:Evaluation Frontend-Framework}

Wie in den Rahmenbedingungen (Kap. \ref{Einführung:Rahmenbedingungen, Umfeld, Definitionen, Abgrenzungen}) festgehalten stehen für die Technologie der Web-Applikation (Frontend) die Optionen React~\cite{react} oder Vue.js~\cite{vuejs} zur Auswahl.
Die Wahl viel auf diese zwei Frameworks, da sich beide momentan schnell verbreiten und zum Design von \ac{SPA} ausgelegt sind, was für unsere Zwecke zum Darstellen einer Karte gut geeignet ist.

Vue.js und React werden im Folgenden anhand mehrerer Kriterien verglichen.
Als Demonstration wird mit beiden Varianten eine identische kleine Web-Applikation entwickelt.
Diese besteht aus einer Web-Karte und einem kleinen Kontrollelement, um einen zusätzlichen Layer einzublenden.
Der Code ist unter~\cite{github:playground} zu finden.

Zum Schluss wird ein Fazit gezogen und entschieden, welches Framework für das Frontend eingesetzt wird.

\subsubsection{Funktionsumfang}
\label{Analyse Framework:Funktionsumfang}

Sowohl React als auch Vue.js bauen auf sehr ähnlichen Prinzipien auf.
Applikationen werden in einzelne Komponenten zerteilt und aufeinander aufgebaut.
In einer Komponente is jeweils die Darstellung (Template) und die UI-Logik miteinander gekoppelt.
Komponenten halten intern Daten, die von einer anderen Komponente bedingt verändert werden können.

Ein Unterschied liegt in der Defintion von HTML-Templates.
Vue.js verwendet reines HTML mit einer erweiterten Template-Syntax, ähnlich wie bei herkömmlichen Template-Engines.
React dagegen benutzt JSX~\cite{jsx}, eine spezielle Syntax, bei der Javascript und HTML gemischt werden können.
In einem Zwischenschritt wird dies zu Javascript kompiliert, das schlussendlich HTML-Elemente erstellt.
Dies macht React etwas mächtiger, da die Templates nicht wie bei Vue.js auf vordefinierte Funktionen beschränkt sind, sondern direkt beliebiges Javascript verwendet werden kann.

Durch die grössere Verbreitung von React gibt es dafür mehr Erweiterungen und eine grössere Community.
Beide Frameworks sind aber schlank gehalten und bieten nur die Kernfunktionalitäten.
Erweiterte Komponenten wie Routing oder State-Management sind in andere Frameworks ausgelagert.

Insgesamt haben React und Vue.js einen sehr ähnlichen Funktionsumfang und verwenden fast identische Prinzipien.

\subsubsection{Integration Leaflet}
\label{Analyse Framework:Integration Leaflet}

Für das Einfügen einer Web-Karte mit Raster-Kacheln in eine Web-Applikation hat sich die Library \emph{Leaflet}~\cite{leaflet} bewährt.
Es wird bewertet, wie einfach sich Leaflet in React und Vue.js einbinden lässt.

Für React bietet \emph{react-leaflet}~\cite{react-leaflet} die Funktionalität von Leaflet an, das Equivalent für Vue.js bildet \emph{Vue2Leaflet}~\cite{vue2leaflet}.

Beide Komponenten bieten eine rudimentäre Dokumentation an, die Bedienung wird aber erst mit den zahlreichen Beispielen klar.
Von der Funktionalität her ist kein grosser Unterschied festzustellen.
\emph{Vue2Leaflet} bildet zwar nicht die komplette Leaflet-Library ab, die für diese Arbeit benötigten Funktionen sind aber alle vorhanden.


\subsubsection{Integration Vector Tiles}
\label{Analyse Framework:Integration Vector Tiles}

Vector Tiles sind Kacheln, die Ausschnitte aus Karten als Vektoren darstellen, anstatt wie bei Raster-Kacheln mit simplen Bildern~\cite{geometalab_vectortiles}.
Dies hat unter anderem den Vorteil, dass nicht bei jeder Zoomstufe ein neues Bild geladen werden muss, sondern die Vektoren beliebig vergrössert werden können.

Für die Integration solcher Vector Tiles in Webkarten hat die Firma Mapbox ein Format spezifiziert und mit \emph{Mapbox GL JS} eine Javascript-Library veröffentlicht, um diese einzubinden.~\cite{mapbox_gl_js}
Diese Library bietet deutlich mehr Möglichkeiten, die Darstellung von Kacheln individuell anzupassen.
Die Vektor-Daten selbst sind bei Mapbox bis zu einem gewissen Abrufvolumen kostenlos erhältlich.
Es können aber auch andere Datenquellen, wie z.B. von OpenMapTiles~\cite{openmaptiles}, verwendet werden.

Für die Integration mit React und Vue.js gibt es mehrere vorgefertigte Komponenten von Dritten.
Wir haben jeweils die beliebtesten (Anzahl Github-Stars) davon angeschaut.
Für React ist dies \emph{react-mapbox-gl}~\cite{react_mapbox_gl}, für Vue.js \emph{vue-mapbox-gl}~\cite{vue_mapbox_gl}.

Die React-Komponente ist gut dokumentiert und relativ einfach zu benutzen.
Dagegen ist die Vue.js-Komponente nur ein einfacher Wrapper um die API von \emph{Mapbox GL}.
So kann z.B. ein zusätzlicher Layer nicht direkt im Template eingebunden werden, sondern muss über Event-Handling nach dem Lader der Karte hinzugefügt werden.
Dies macht das dynamische Hinzufügen und Entfernen von Karten-Elementen, was wir in unserer Applikation viel benötigen, sehr mühsam.

\subsubsection{Bedienung}
\label{Analyse Framework:Bedienung}



\subsubsection{Tooling}
\label{Analyse Framework:Tooling}

Da sowohl React wie auch Vue.js auf NodeJS aufsetzen und den Packetmanager NPM benutzen, ist das Tooling sehr ähnlich.

Um eine neue Applikation zu erstellen, bietet React das Tool \emph{create-react-app}~\cite{create_react_app} an.
Beim Ausführen kann man angeben, welche zusätzlichen Libraries (z.B. für Unit-Testing) man verwenden möchte.
Während der Entwicklung kann ausserdem ein lokaler Server gestartet werden, der jeweils automatisch das Browser-Fenster neu ladet, wenn sich der Source-Code geändert hat.

Vue.js bietet mit \emph{vue-cli}~\cite{vue_cli} ein ähnliches Tool an.
Um eine neue Applikation aufzusetzen, kann zwischen verschiedenen Templates gewählt werden.
Es gibt verschiedene Templates für z.B. unterschiedliche Grössen der geplanten Applikation.
Während der Entwicklung bietet \emph{vue-cli} die praktisch identischen Möglichkeiten wie das Tool für React.

Insgesamt kann zwischen den beiden Varianten in Sachen Tooling praktisch keinen Unterschied festgestellt werden.


\subsubsection{Auswertung}
\label{Analyse Framework:Auswertung}

Für die Bewertung wird jedem Kriterium einen Wert von 1 bis 6 zugewiesen, wobei 6 die Bestnote ist.
Jedes Kriterium wird anhand der Relevanz und Wichtigkeit in unserem Projekt gewichtet.

\begin{longtable}{r l p{3cm} l l}
    \toprule
    \textbf{ID} &
        \textbf{Kriterium} 
                                                    & \textbf{Relatives Gewicht (0-1)} &
                                                                                  \textbf{Vue.js} &
                                                                                                  \textbf{React} \\
    1     & \nameref{Analyse Framework:Funktionsumfang}           & 0.1           & x.xx / x.xx   & x.xx / x.xx  \\
    2     & \nameref{Analyse Framework:Integration Leaflet}       & 0.3           & x.xx / x.xx   & x.xx / x.xx  \\
    3     & \nameref{Analyse Framework:Integration Vector Tiles}  & 0.1           & x.xx / x.xx   & x.xx / x.xx  \\
    3     & \nameref{Analyse Framework:Bedienung}                 & 0.2           & x.xx / x.xx   & x.xx / x.xx  \\
    3     & \nameref{Analyse Framework:Tooling}                   & 0.3           & x.xx / x.xx   & x.xx / x.xx  \\
    \bottomrule
    \multicolumn{3}{l}{\textbf{Total}}                                            & \textbf{x.xx / x.xx}
                                                                                            & \textbf{x.xx / x.xx} \\
    \caption{Resultat der Frontend-Framework-Evaluation}
    \label{table:Resultat der Frontend-Framework-Evaluation}
\end{longtable}


\subsubsection{Einschränkungen}
\label{Analyse:Einschränkungen}

%TODO
