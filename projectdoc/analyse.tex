
\section{Analyse}
\label{Analyse}

%TODO

\subsection{Evaluation Frontend-Framework}
\label{Analyse:Evaluation Frontend-Framework}

Wie in den Rahmenbedingungen (Kap. \ref{Einführung:Rahmenbedingungen, Umfeld, Definitionen, Abgrenzungen}) festgehalten stehen für die Technologie der Web-Applikation (Frontend) die Optionen React~\cite{react} oder Vue.js~\cite{vuejs} zur Auswahl.
Die Wahl viel auf diese zwei Frameworks, da sich beide momentan schnell verbreiten und zum Design von \ac{SPA} ausgelegt sind, was für unsere Zwecke zum Darstellen einer Karte gut geeignet ist.

Vue.js und React werden im Folgenden anhand mehrerer Kriterien verglichen.
Zum Schluss wird ein Fazit gezogen und entschieden, welches Framework für das Frontend eingesetzt wird.

\subsubsection{Funktionsumfang}
\label{Analyse Framework:Funktionsumfang}

Sowohl React als auch Vue.js bauen auf sehr ähnlichen Prinzipien auf.
Applikationen werden in einzelne Komponenten zerteilt und aufeinander aufgebaut.
In einer Komponente is jeweils die Darstellung (Template) und die UI-Logik miteinander gekoppelt.
Komponenten halten intern Daten, die von einer anderen Komponente bedingt verändert werden können.

Ein Unterschied liegt in der Defintion von HTML-Templates.
Vue.js verwendet reines HTML mit einer erweiterten Template-Syntax, ähnlich wie bei herkömmlichen Template-Engines.
React dagegen benutzt JSX~\cite{jsx}, eine spezielle Syntax, bei der Javascript und HTML gemischt werden können.
In einem Zwischenschritt wird dies zu Javascript kompiliert, das schlussendlich HTML-Elemente erstellt.
Dies macht React etwas mächtiger, da die Templates nicht wie bei Vue.js auf vordefinierte Funktionen beschränkt sind, sondern direkt beliebiges Javascript verwendet werden kann.

Durch die grössere Verbreitung von React gibt es dafür mehr Erweiterungen und eine grössere Community.
Beide Frameworks sind aber schlank gehalten und bieten nur die Kernfunktionalitäten.
Erweiterte Komponenten wie Routing oder State-Management sind in andere Frameworks ausgelagert.

Insgesamt haben React und Vue.js einen sehr ähnlichen Funktionsumfang und verwenden fast identische Prinzipien.

\subsubsection{Integration Web-Karte}
\label{Analyse Framework:Integration Web-Karte}

Für das Einfügen einer Web-Karte in die Web-Applikation hat sich die Library \emph{Leaflet}~\cite{leaflet} bewährt.
Es wird bewertet, wie einfach sich Leaflet in React und Vue.js einbinden lässt.

Für React bietet \emph{react-leaflet}~\cite{react-leaflet} die Funktionalität von Leaflet an, das Equivalent für Vue.js bildet \emph{Vue2Leaflet}~\cite{vue2leaflet}.

Beide Komponenten bieten eine rudimentäre Dokumentation an, die Bedienung wird aber erst mit den zahlreichen Beispielen klar.
Von der Funktionalität her ist kein grosser Unterschied festzustellen.
\emph{Vue2Leaflet} bildet zwar nicht die komplette Leaflet-Library ab, die für diese Arbeit benötigten Funktionen sind aber alle vorhanden.


\subsubsection{Bedienung}
\label{Analyse Framework:Bedienung}

\subsubsection{Tooling}
\label{Analyse Framework:Bedienung}


\subsubsection{Resultat}
\label{Analyse Framework:Resulat}

%TODO


\subsubsection{Einschränkungen}
\label{Analyse:Einschränkungen}

%TODO
