
\section{Anforderungsspezifikation}
\label{Anforderungsspezifikation}

\subsection{Use Cases}
\label{Anforderungsspezifikation:Use Cases}

Im folgenden sind die funktionalen Anforderungen an ... mit all seinen Komponenten, welche im Kapitel \ref{Architektur} aufgeführt sind, als Use Cases im Brief-Format beschrieben. Zur Übersicht ist das Use Case Diagramm in Abbildung \code{TODO} zu betrachten.

%TODO add use case diagram

\subsubsection{Aktoren}
\label{Use Cases:Aktoren}

\begin{table}[h]
    \centering
    \caption{Aktoren}
    \label{aktoren}
    \begin{tabular}{ll}
        \textbf{Aktor} & \textbf{Beschreibung und Interessen}                                                                                                                                                                                     \\
        \textbf{User}  & \begin{tabular}[c]{@{}l@{}}Ein User ist ...\end{tabular}                                          \\
        \textbf{Admin} & \begin{tabular}[c]{@{}l@{}}Ein Admin ist ...\end{tabular}
    \end{tabular}
\end{table}

\subsubsection{UC01: TODO}
\label{Use Cases:UC01}

Aktoren: \emph{User}

Include: \nameref{Use Cases:UC02}

%TODO

\subsubsection{UC02: TODO}
\label{Use Cases:UC02}

Aktoren: \emph{User}

%TODO


\subsection{Nicht-funktionale Anforderungen}
\label{Anforderungsspezifikation:Nicht-funktionale Anforderungen}

\subsubsection{NFA01: TODO}
\label{NFA:NFA01}

%TODO

