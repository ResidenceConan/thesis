
\section{Implementation}
\label{Implementation}

In diesem Kapitel wird die Implementation der einzelnen Container beschrieben, wie sie im Kapitel \ref{Architektur:Container} definiert wurden.

\subsection{ÖV-Güteklassen 2018 Generator}
\label{Implementation:ÖV-Güteklassen 2018 Generator}

\subsubsection{Mapping Verkehrsmittelgruppe und GTFS Route Type}
\label{ÖV-Güteklassen 2018 Generator:Mapping Verkehrsmittelgruppe und GTFS Route Type}

Die öffentlichen Verkehrsmittel werden in drei Verkehrsmittelgruppen gruppiert (siehe Kapitel \ref{Berechnungsmethodik OeVGK18:Art der Verkehrsmittel}).
Wie in Kapitel \ref{subsystem:GTFS} beschrieben, werden die Fahrplandaten im \acs{GTFS}-Format~\cite{gtfs_spec} gehalten. 
Es werden dabei 8 Verkehrsmittel-Typen definiert.
In Tabelle \ref{table:Mapping GTFS Route Type Verkehrsmittelgruppe} ist ein Mapping der definierten GTFS Route Typen und der Verkehrsmittelgruppen ersichtlich.

\begin{longtable}[ht]{l l}
        \midrule
        \textbf{GTFS Route Type} 
                                & \textbf{Verkehrsmittelgruppe}\\
        0 (Tram, Streetcar, Light rail)
                                & B\\
        1 (Subway, Metro)
                                & B\\
        2 (Rail)
                                & A\\
        3 (Bus)
                                & B\\
        4 (Ferry)
                                & B\\
        5 (Cable car)
                                & (C)\\
        6 (Gondola, Suspended cable car)
                                & C\\
        7 (Funicular)
                                & C\\            
        \bottomrule
    \caption{Mapping GTFS Route Type Verkehrsmittelgruppe}
    \label{table:Mapping GTFS Route Type Verkehrsmittelgruppe}
\end{longtable}
