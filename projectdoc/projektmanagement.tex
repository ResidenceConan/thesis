% Aufwandschätzung, Zeitplan, Projektplan

\section{Projektmanagement}
\label{Projektmanagement}

\subsection{Vorgehen}
\label{Projektmanagement:Vorgehen}

Für die Bachelorarbeit wurde das agile Vorgehen SCRUM in Kombination mit \ac{RUP} gewählt.
Gründe für diese Entscheidung sind, dass das agile Vorgehen der noch zu Beginn offenen Spezifikation der Güteklassen entgegenkommt, welche dann iterativ finalisiert werden kann und dass der theoretische Fokus der Arbeit so besser gehandhabt und schneller reagiert werden kann.
Die wöchentlichen Besprechungen und Reviews mit dem Betreuer ist ein weiterer Grund für diese Entscheidung.
Die Kombination mit \ac{RUP} ermöglicht es, dass Projekt in einzelne Phasen aufzuteilen, um so das Ziel und die Zeit nicht aus den Augen zu verlieren.

%TODO

\subsubsection{Entwicklung}
\label{Vorgehen:Entwicklung}

Der Source-Code der Implementation wie auch diese Arbeit wird mit Git verwaltet und ist auf Github abgelegt.
Die Entwicklung und das Dokumentieren erfolgt nach dem Github-Flow.
Der Master-Branch ist auf allen Repositories während der ganzen Zeit gesperrt, so dass er nur über Pull-Requests bearbeitet werden kann.
Für jede User-Story wird ein Branch erstellt.
Ist die User-Story implementiert, wird ein Pull-Request erstellt und dem anderen Projekt-Mitglied zum Review übergeben.
Wird der Pull-Request akzeptiert, wird der Feature-Branch in den Master gemerged.
Dieses Vorgehen hat den Vorteil, dass alle Änderungen, welche in den Master gelangen, ein Review durchlaufen müssen und so die Qualität hochgehalten werden kann.


\subsection{Zeitplanung}
\label{Projektmanagement:Zeitplanung}

Die Arbeitspakte und Zeit wird mithilfe von Jira verwaltet.
Für alle Tätigkeiten werden User-Stories im Backlog erfasst, priorisiert und geschätzt.
Die Schätzung der User-Stories erfolgte mit Story Points.
Die Arbeitszeitverbuchung wurde auf Arbeitspaket-Stufe mit Stunden gemacht.

\subsubsection{Phasen / Iterationen und Meilensteine}
\label{Zeitplanung:Phasen / Iterationen und Meilensteine}

Die Studienarbeit wird in die \ac{RUP}-Phasen (Inception, Elaboration, Construction, Transition) aufgeteilt.
Dabei wird jedoch eine von der gängigen Norm abweichende Aufteilung gewählt.
Durch den theoretischen Fokus der Arbeit wird der Elaboration das grösste Zeitbudget zugeordnet.
Dies ist auch der Grund, warum mit einwöchigen Sprints gearbeitet wird.
Es wird zusätzlich vom Standard-\ac{RUP}-Prozess abgewichen.
In den letzten zwei Wochen des Projekts wird der Aufwand pro Sprint verdoppelt (40h pro Person), da in dieser Zeit Vollzeit am Projekt gearbeitet werden kann.

\begin{landscape}
\begin{longtable}{l p{6.5cm} p{6.5cm} p{6.5cm}}
        \toprule
        \textbf{Sprint}
                                & \textbf{Sprint 1}
                                & \textbf{Sprint 2}
                                & \textbf{Sprint 3} \\

        \midrule
        \textbf{Phase}
                                & Inception
                                & Elaboration
                                & Elaboration \\

        \textbf{Milestones}
                                & \textit{Projektantrag genehmigt}
                                & \textit{Projektplan erstellt, Scope abgesteckt}
                                & \textit{Stand der Technik evaluiert}  \\

        \textbf{Inhalt}
                                & \begin{enumerate}[noitemsep]
                                    \item Projektantrag erstellen
                                    \item Grobplanung erstellen
                                    \item LaTex und CI aufsetzen
                                \end{enumerate}
                                & \begin{enumerate}[noitemsep]
                                    \item Projektplan erstellen
                                    \item FA/NFA erarbeiten
                                    \item Abgrenzungen definieren
                                \end{enumerate}
                                & \begin{enumerate}[noitemsep]
                                    \item in Norm 640 290 einarbeiten
                                    \item aktuelle Berechnungsmethodik aufschlüsseln
                                    \item Fremdsysteme \& Datenquellen eruieren
                                \end{enumerate}\\

        \toprule
        \textbf{Sprint}
                                & \textbf{Sprint 4}
                                & \textbf{Sprint 5}
                                & \textbf{Sprint 6} \\
        \midrule
        \textbf{Phase}
                                & Elaboration
                                & Elaboration
                                & Elaboration \\

        \textbf{Milestones}
                                & \textit{Stand der Technik evaluiert}
                                & \textit{technische Machbarkeit geprüft}
                                & \textit{Spezifikation umsetzungsbereit}  \\

        \textbf{Inhalt}
                                & \begin{enumerate}[noitemsep]
                                    \item Anbindung Fremdsysteme \& Datenquellen evaluieren
                                    \item Framework für Frontend evaluieren
                                    \item Auslieferung Kartendaten an Frontend evaluieren
                                \end{enumerate}
                                & \begin{enumerate}[noitemsep]
                                    \item in PostGIS einarbeiten
                                    \item Machbarkeitsanalyse pgRouting durchführen
                                \end{enumerate}
                                & \begin{enumerate}[noitemsep]
                                    \item Verbesserungen der Berechnungsmethoden erarbeiten
                                    \item Spezifikation erstellen
                                \end{enumerate}  \\

        \pagebreak
        \toprule
        \textbf{Sprint}
                                & \textbf{Sprint 7}
                                & \textbf{Sprint 8}
                                & \textbf{Sprint 9} \\

        \midrule
        \textbf{Phase}
                                & Elaboration
                                & Elaboration
                                & Construction \\

        \textbf{Milestones}
                                & \textit{Spezifikation umsetzungsbereit}
                                & \textit{End of Elaboration}
                                & \textit{Spezifikation implementiert}  \\

        \textbf{Inhalt}
                                & \begin{enumerate}[noitemsep]
                                    \item Verbesserungen der Berechnungsmethoden erarbeiten
                                    \item Spezifikation erstellen
                                \end{enumerate}
                                & \begin{enumerate}[noitemsep]
                                    \item Architektur definieren
                                    \item Infrastruktur aufsetzen
                                    \item Frontend Design Entwurf erstellen
                                \end{enumerate}
                                & \begin{enumerate}[noitemsep]
                                    \item Konfigurationsmöglichkeiten definieren
                                    \item Spezifikation umsetzen
                                \end{enumerate} \\

        \toprule
        \textbf{Sprint}
                                & \textbf{Sprint 10}
                                & \textbf{Sprint 11}
                                & \textbf{Sprint 12} \\

        \midrule
        \textbf{Phase}
                                & Construction
                                & Construction
                                & Construction \\

        \textbf{Milestones}
                                & \textit{Spezifikation implementiert}
                                & \textit{Spezifikation implementiert}
                                & \textit{Spezifikation implementiert}  \\

        \textbf{Inhalt}
                                & \begin{enumerate}[noitemsep]
                                    \item Spezifikation umsetzen
                                \end{enumerate}
                                & \begin{enumerate}[noitemsep]
                                    \item Spezifikation umsetzen
                                \end{enumerate}
                                & \begin{enumerate}[noitemsep]
                                    \item Berechnung automatisieren
                                \end{enumerate} \\

        \pagebreak
        \toprule
        \textbf{Sprint}
                                & \textbf{Sprint 13}
                                & \textbf{Sprint 14}
                                & \textbf{Sprint 15}\\

        \midrule
        \textbf{Phase}
                                & Construction
                                & Construction
                                & Construction\\

        \textbf{Milestones}
                                & \textit{Web-Applikation entwickelt}
                                & \textit{Web-Applikation entwickelt}
                                & \textit{Web-Applikation entwickelt}\\

        \textbf{Inhalt}
                                & \begin{enumerate}[noitemsep]
                                    \item Backend für Auslieferung der Kartendaten erstellen
                                    \item Frontend-Entwurf umsetzen
                                \end{enumerate}
                                & \begin{enumerate}[noitemsep]
                                    \item Backend für Auslieferung der Kartendaten erstellen
                                \end{enumerate}
                                & \begin{enumerate}[noitemsep]
                                    \item Frontend fertig stellen
                                \end{enumerate}\\

        \toprule
        \textbf{Sprint}
                                & \textbf{Sprint 16}
                                & \\
                                & \\

        \midrule
        \textbf{Phase}
                                & Transition
                                & \\
                                & \\

        \textbf{Milestones}
                                & \textit{BA abgegeben}
                                & \\
                                &  \\

        \textbf{Inhalt}
                                & \begin{enumerate}[noitemsep]
                                    \item Präsentation erstellen
                                    \item Plakat erstellen
                                    \item Arbeit finalisieren
                                    \item Arbeit abgeben

                                \end{enumerate}
                                & \\
                                & \\
        \bottomrule
    \caption{Phasen / Iterationen und Meilensteine}
    \label{table:Phasen / Iterationen und Meilensteine}
\end{longtable}
\end{landscape}


\subsection{Risiken}
\label{Projektmanagement:Risiken}

%TODO


\subsubsection{Umgang mit Risiken}
\label{Risiken:Umang mit Risiken}

%TODO

\subsubsection{Konsequenz}
\label{Risiken:Konsequenz}

%TODO
