
\cleardoublepage
\section{Softwaredokumentation}
\label{Softwaredokumentation}

Grundsätzlich besteht die Software aus den Komponenten \gls{ÖV-Güteklassen} 2018-Generator (siehe Kapitel \ref{Implementation:ÖV-Güteklassen 2018 Generator}), Backend (siehe Kapitel \ref{Implementation:Backend}) und Web-App (siehe Kapitel \ref{Implementation:Web-App}).

Die Installations- und Benutzeranleitungen zu den Komponenten sind in den jeweiligen Github-Repositories gehalten.

Für einen raschen Einstieg sind hier die Verweise auf die einzelnen Anleitungen der verschiedenen Komponenten aufgeführt.

\subsection{ÖV-Güteklassen 2018 Generator}
\label{Softwaredokumentation: ÖV-Güteklassen 2018 Generator}

Die Installations- und Benutzeranleitungen für den \emph{\gls{ÖV-Güteklassen} 2018-Generator} befindet sich auf Github~\cite{github:oevgk18-generator}.

\subsection{Backend}
\label{Softwaredokumentation: Backend}

Die Installations- und Benutzeranleitungen für das \emph{Backend} befindet sich auf Github~\cite{github:backend}.

Die Web-\acs{API}, welche das Backend exponiert, kann mit ReDoc (OpenAPI/Swagger-generated API Reference Documentation)~\cite{oevgk18-backend-api-spec} oder mit dem klassischen Swagger UI~\cite{oevgk18-backend-api-swaggerui} betrachtet werden.

\subsection{Web-App}
\label{Softwaredokumentation: Web-App}

Die Installations- und Benutzeranleitungen für das \emph{Web-App} befindet sich auf Github~\cite{github:web-app}.
Der \emph{Tile-Server} sowie der \emph{Tile-Converter} sind ebenfalls in dieser Anleitung beschrieben.
