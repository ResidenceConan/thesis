
\section{Softwaredokumentation}
\label{Softwaredokumentation}

In diesem Kapitel wird die Bedienung der Software beschrieben.
Diese besteht aus dem \nameref{Implementation:ÖV-Güteklassen 2018 Generator}, dem Backend und dem Frontend.

Mit dem Generator werden zuerst die \ac{ÖV}-Güteklassen erzeugt.
Mit dem Resultat davon wird das Backend, ein HTTP-Server, gestartet.
Dieser wiederum wird vom Frontend als Datenquelle gebraucht, um die \acs{ÖV}-Güteklassen zu visualisieren.

\subsection{ÖV-Güteklassen 2018 Generator}
\label{Softwaredokumentation: ÖV-Güteklassen 2018 Generator}

\subsubsection{Vorbereitung}
\label{Softwaredokumentation:Vorbereitung}
Als Vorbereitung für das Generieren der \ac{ÖV}-Güteklassen wird eine Datenbank aufgesetzt, die alle zur Berechnung nötigen Daten, Schemas und Indices enthält.
Um diesen Prozess möglichst automatisiert und reproduzierbar auszuführen, wurden entsprechende Docker-Container und Scripts vorbereitet.
Vorgängig müssen dazu die Terraindaten der Schweiz als TIF besorgt werden.
% TODO werden OSM-Daten zum Schluss auch automatisch herunter geladen?
Die restlichen Daten werden werden automatisch während dem Setup vom Internet bezogen.

Die genaue Anleitung zur Vorbereitung der Datenbank ist auf Github~\cite{github_docker_setup} verfügbar.

