% Manuelle und automatische Tests

\section{Tests}
\label{Tests}

\subsection{Strategie}
\label{Tests:Strategie}

\subsubsection{Web-Applikation}
\label{tests:Web-Applikation}

\paragraph{Unit-Tests}~\\
Für das in React geschriebene Frontend werden Unit-Tests mit \emph{Jest}~\cite{jest} geschrieben.
Dies ist ein von Facebook entwickeltes Testing-Framework, das speziell (aber nicht exklusiv) für React konzipiert wurde.

% TODO: Snapshot testing erwähnen, wenn wir das gebrauchen


\paragraph{Type-Checking}~\\
Da JavaScript eine schwach typisierte Sprache ist, macht es Sinn, während der Entwicklung statisches Type-Checking zu verwenden.
So können bereits zu Beginn des Entwicklungsprozesses Fehler mit Typisierungen vermieden werden.

Dafür bietet sich die Library \emph{Flow}~\cite{flow} an, die wie React ebenfalls von Facebook entwickelt wird.
Mit dieser ist es möglich, bei allen Deklarationen die Typen anzugeben.
Kontinuierlich wird dann geprüft, dass die Typisierungen korrekt verwendet werden.



\subsubsection{Healthchecks}
\label{Tests:Healthchecks}

%TODO

\subsection{Fazit}
\label{Tests:Fazit}

%TODO
