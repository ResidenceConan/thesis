\cleardoublepage
\section{Tests}
\label{Tests}

\subsection{ÖV-Güteklassen 2018 Generator}
\label{Tests:ÖV-Güteklassen 2018 Generator}

Der Generator für die \gls{ÖV-Güteklassen} 2018 wurde in Python implementiert.
Für Unit-Tests hat sich dafür in vorhergehenden Projekten das Testing-Framework \emph{pytest}~\cite{pytest} bewährt.

Die Architektur wurde so gestaltet, dass die Business-Schicht keine direkte Abhängigkeit auf die Integrations-Schicht hat, die für die Kommunikation mit der Datenbank verantwortlich ist, und so isoliert getestet werden kann.
Die Abhängigkeiten wurden stattdessen dynamisch in einer "`Registry"', analog zu einer Dependency Injection, gehalten.
Dies erlaubte es uns, in den Unit-Tests die Module in der Integrations-Schicht durch Mocks auszutauschen.

Da einige Logik der Berechnung in \glspl{Stored Procedure} auf der Datenbank implementiert wurden, wäre es wünschenswert, diese ebenfalls automatisiert zu testen.
Zwar gibt es dafür Ansätze wie \emph{pgTAP}~\cite{pgTAP}, um Unit-Tests für Datenbank-Funktionen zu schreiben, bei uns war aber neben der Logik auch die Performance ein entscheidender Faktor.
Diese konnte nur manuell auf dem kompletten Datensatz der Schweiz getestet werden, denn erst dann wurde klar, welche Indizes benötigt werden und wie die Datenbank eine bestimmte Query ausführt.

Vorstellbar wären hingegen Integrations-Tests, die auf einem Test-Set die kompletten \gls{ÖV-Güteklassen} berechnen.
Die Schwierigkeit daran ist das aufwändige Setup mit Fahrplan-, Routing- und Höhendaten, die alle in einer laufenden Datenbank aufgesetzt werden müssten.
Dabei wäre es natürlich wünschenswert, das ganze Setup automatisiert in einer Continuous Integration Umgebung zu betreiben.

\subsection{Backend}
\label{Tests:Backend}

Für das Backend wurde das automatische Testing auf minimale Tests beschränkt, die prüfen, dass der Server auf API-Requests keine unerwartete Antworten liefert.
Während der Entwicklung wurde das Backend immer schlanker.
Wurden zuerst die kompletten \gls{GeoJSON}-Daten für die Web-Applikation ausgeliefert, wird seit dem Wechsel zu Vector Tiles nur noch ein JSON mit Metadaten zu den Parametern eingelesen und über ein API angeboten.
Dementsprechend ist neben dem Exponieren von Metadaten keine mehr Logik vorhanden.

\subsection{Web-Applikation}
\label{tests:Web-Applikation}

\paragraph{Unit-Tests}~\\
Für das in React geschriebene Frontend wurde ursprünglich geplant, Unit-Tests mit \emph{Jest}~\cite{jest} zu schreiben.
Allerdings kam schon früh zur Entwicklung das Problem auf, dass die \emph{Mapbox GL} Komponente, die für die Basiskarte verwendet wird, nicht in einem Test geladen werden kann, weil dafür ein Browser benötigt wird.
Da die einzige Logik in der Applikation in der Verarbeitung der Metadaten vom Backend besteht, wären reine Unit-Tests aufwändig.
Sinnvoller wäre es, ein Integration-Test zusammen mit dem Backend-API zu machen.
Dies erfordert aber ein aufwändiges Setup, um dies automatisiert mit einer Continuous Integration zu verbinden.
Dieser Aufwand wäre aufgrund der Anforderungen der Web-Applikation nicht gerechtfertigt.

\paragraph{Type-Checking}~\\
Da JavaScript eine schwach typisierte Sprache ist, macht es Sinn, während der Entwicklung statisches Type-Checking zu verwenden.
So können bereits zu Beginn des Entwicklungsprozesses Fehler mit Typisierungen vermieden werden.

Dafür bietet sich die Library \emph{Flow}~\cite{flow} an, die wie React ebenfalls von Facebook entwickelt wird.
Mit dieser ist es möglich, bei allen Deklarationen die Typen anzugeben.
Kontinuierlich wird dann geprüft, dass die Typisierungen korrekt verwendet werden.
