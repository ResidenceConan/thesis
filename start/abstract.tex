% Der Abstract richtet sich an den Spezialisten auf dem entsprechenden Gebiet
% und beschreibt daher in erster Linie die (neuen, eigenen) Ergebnisse und
% Resultate der Arbeit. Es umfasst nie mehr als eine Seite, typisch sogar nur
% etwa 200 Worte (etwa 20 Zeilen). Es sind keine Bilder zu verwenden.

\chapter*{Abstract}\addcontentsline{toc}{chapter}{Abstract}
ÖV-Güteklassen werden für die Beurteilung der Erschliessung mit dem öffentlichen Verkehr verwendet.
Die heute anerkannte Spezifikation des Bundesamt für Raumentwicklung (ARE) der ÖV-Güteklassen basiert auf einer inzwischen ersetzten Schweizer Norm aus dem Jahre 1993.
Diverse Kantone haben in Eigeninitiativen Anpassungen daran vorgenommen, um kantonalen Gegebenheiten Gerecht zu werden.
Die Implementationen dieser sind überholt und werden den aktuellen technischen Möglichkeiten nicht gerecht.
So führen einige Kantone die Berechnung dieser in Handarbeit durch.
Ebenfalls wird das Einzugsgebiet mit Luftlinien berechnet und der Topografie wird nicht konsequent in die Berechnung einbezogen.
Die kantonalen Eigenlösungen zeigen, dass kein Konsens in der aktuellen Lösung vorhanden ist.

Die Spezifikation ÖV-Güteklassen 2018 kombiniert die Learnings der kantonalen Lösungen und verbessert diese mit den aktuellen technischen Möglichkeiten mit dem Ziel eine allgemeingültige Spezifikation für die Schweiz zu erstellen.
Dadurch kann das Einzugsgebiet auf einem Routing-Netzwerk mit Isochronen berechnet werden und die Topografie wird konsequent in Kombination mit dem  hoch aufgelösten digitalem Terrainmodell von swisstopo berücksichtigt.

%TODO results

\cleardoublepage

\chapter*{Abstract}

%TODO