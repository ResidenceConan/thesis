% Der Abstract richtet sich an den Spezialisten auf dem entsprechenden Gebiet
% und beschreibt daher in erster Linie die (neuen, eigenen) Ergebnisse und
% Resultate der Arbeit. Es umfasst nie mehr als eine Seite, typisch sogar nur
% etwa 200 Worte (etwa 20 Zeilen). Es sind keine Bilder zu verwenden.

\chapter*{Abstract}\addcontentsline{toc}{chapter}{Abstract}
ÖV-Güteklassen werden für die Beurteilung der Erschliessung mit dem öffentlichen Verkehr verwendet.
Die heute anerkannte Spezifikation des Bundesamt für Raumentwicklung (ARE) der ÖV-Güteklassen basiert auf einer inzwischen ersetzten Schweizer Norm aus dem Jahre 1993.
Diverse Kantone haben in Eigeninitiativen Anpassungen daran vorgenommen, um kantonalen Gegebenheiten Gerecht zu werden.
Die Implementationen dieser sind überholt und werden den aktuellen technischen Möglichkeiten nicht gerecht.
So führen einige Kantone die Berechnung dieser in Handarbeit durch.
Ebenfalls wird das Einzugsgebiet mit Luftlinien berechnet und der Topografie wird nicht konsequent in die Berechnung einbezogen.
Die kantonalen Eigenlösungen zeigen, dass kein Konsens in der aktuellen Lösung vorhanden ist.

Die Spezifikation ÖV-Güteklassen 2018 kombiniert die Learnings der kantonalen Lösungen und verbessert diese mit den aktuellen technischen Möglichkeiten mit dem Ziel eine allgemeingültige Spezifikation für die Schweiz zu erstellen.
In der Spezifikation wird nun vorausgesetzt, dass das Einzugsgebiet auf einem Routing-Graphen berechnet wird.
Ebenfalls muss die Topografie konsequent berücksichtigt werden.
Diese Anforderung wird mit der Berechnung von Leistungskilometern erreicht.
Die Berechnung des Intervall eine Haltestelle wurde überdacht und eine neue Formel für die Bestimmung definiert.

Der ÖV-Güteklassen 2018 Generator setzt die erstellte Spezifikation um.
So werden für das Einzugsgebiet mithilfe von pgRouting auf einem Routing-Graphen mit Isochronen berechnet.
Dadurch hat man ein viel detaillierteres Verständnis über die Erreichbarkeit der Haltestelle.
Durch den Einsatz eines hoch aufgelösten digitalem Terrainmodell von swisstopo wird die Topografie konsequent in die Berechnung miteinbezogen und ein Haltestelle auf einem schwer begehbaren Gelände hat somit ein kleineres Einzugsgebiet.
Der Generator erlaubt die automatisierte Berechnung der ÖV-Güteklassen für verschiedene Stichtage.

Zur Auswertung werden die aufgrund der neuen Spezifikation berechneten ÖV-Güteklassen in einer Webapplikation dargestellt.
Dabei können diese visuell den ÖV-Güteklassen des Bundesamt für Raumentwicklung gegenübergestellt werden.



%TODO results

\cleardoublepage

\chapter*{Abstract}

%TODO