% Der Abstract richtet sich an den Spezialisten auf dem entsprechenden Gebiet
% und beschreibt daher in erster Linie die (neuen, eigenen) Ergebnisse und
% Resultate der Arbeit. Es umfasst nie mehr als eine Seite, typisch sogar nur
% etwa 200 Worte (etwa 20 Zeilen). Es sind keine Bilder zu verwenden.

\chapter*{Abstract}\addcontentsline{toc}{chapter}{Abstract}
ÖV-Güteklassen werden für die Beurteilung der Erschliessung mit dem öffentlichen Verkehr verwendet.
Die heute anerkannte Spezifikation der ÖV-Güteklassen des Bundesamts für Raumentwicklung (ARE)  basiert auf einer inzwischen ersetzten Schweizer Norm aus dem Jahre 1993.
Diverse Kantone haben in Eigeninitiativen Anpassungen daran vorgenommen, um kantonalen Gegebenheiten gerecht zu werden.
Die Implementationen dieser sind überholt und werden den aktuellen technischen Möglichkeiten nicht gerecht.
So führen einige Kantone die Berechnung dieser in Handarbeit durch.
Ebenfalls wird das Einzugsgebiet mit Luftlinien berechnet und die Topografie nicht konsequent in die Berechnung mit einbezogen.
Die kantonalen Eigenlösungen zeigen, dass noch kein Konsens einer aktuellen Lösung vorhanden ist.

Die neu erarbeiteten ÖV-Güteklassen 2018 kombinieren die Erkenntnisse einiger kantonalen Lösungen und verbessert diese mit den aktuellen technischen Möglichkeiten mit dem Ziel, eine allgemeingültige Spezifikation für die Schweiz zu erstellen.
In der Spezifikation wird nun vorausgesetzt, dass das Einzugsgebiet auf einem Fussgänger-Routing-Graphen berechnet wird.
Ebenfalls soll die Topografie konsequent berücksichtigt werden.
Diese Anforderung wird mit der Berechnung von Leistungskilometern erreicht.

Der ÖV-Güteklassen 2018 Generator setzt die neu erstellte Spezifikation um.
Er erzeugt einen schweizweiten Geodatensatz.
So werden mithilfe von OpenStreetMap-Daten und mit der pgRouting-Software (PostgreSQL) auf einem Fussgänger-Graphen für jedes Einzugsgebiet Isochronen berechnet.
Dadurch erhält man ein adäquates Verständnis über die Erreichbarkeit einer Haltestelle.
Durch den Einsatz des hochaufgelösten digitalen Terrainmodells swissALTI$^{3D}$ von Swisstopo wird die Topografie in die Berechnung mit einbezogen.
Eine Haltestelle auf einem schwer begehbaren Gelände hat somit ein kleineres Einzugsgebiet.
Die Berechnung des Kursintervalls einer Haltestelle wurde überdacht und eine neue Formel definiert.
Der Generator erlaubt die automatisierte Berechnung der ÖV-Güteklassen für verschiedene Stichtage.

Zur Veranschaulichung werden die berechneten ÖV-Güteklassen 2018 in einer
Webapplikation dargestellt.
Dabei können diese den ÖV-Güteklassen des ARE überlagert werden.

ÖV-Güteklassen 2018 geht mit der Zeit und passt sich denn aktuellen technischen Möglichkeiten an.
Durch die Konsolidierung einiger kantonaler Lösungen und Berücksichtigung der genauen Wegführung und Topografie wurde der erste Schritt in Richtung einer schweizweit anerkannter ÖV-Güteklassen-Spezifikation gemacht.

\cleardoublepage

\chapter*{Abstract}

Public transport quality gradings ("`ÖV-Güteklassen"') are used to measure the coverage of public transportation (...).
The currently accepted specification by the Swiss Federal Office for Spatial Development (ARE) is based on a superseded standard from 1993.
Multiple cantons have since made adjustments to the methods to cater to their specific requirements (/ needs?).
The implementation of those are not using the full capabilities of geospatial analysis.
For instance, they are using linear distances for coverage areas (?) and do not incorporate topographic data into the calculation.
The individual approaches from the cantons show that there has been no consensus for a general solution.

This thesis introduces a new specification for public transport quality gradings ("ÖV-Güteklassen 2018") which incorporates approaches from the cantons and combines them with more modern technical capabilities with the aim to create a general solution for Switzerland.
The specification requires the coverage areas to be calculated along a pedestrian routing graph and the topography to be incorporated.

Our software implements the newly created specification.
With the help of the routing engine "`pgRouting"', we create coverage areas along a pedestrian routing graph, which allows for a more detailed understanding of the public transport coverage.
The topography is integrated with a high-resolution digital height model from swisstopo.
This way, a public transport stop on (e.g.) a steep hill gets a smaller area of coverage.
Furthermore, we define a new formula to calculate the interval of a public transport stop.

To evaluate the results of the calculation for the new specification, we created an web application which allows the coverage areas to be examined and compared with the current public transport gradings from ARE.