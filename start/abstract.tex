% Der Abstract richtet sich an den Spezialisten auf dem entsprechenden Gebiet
% und beschreibt daher in erster Linie die (neuen, eigenen) Ergebnisse und
% Resultate der Arbeit. Es umfasst nie mehr als eine Seite, typisch sogar nur
% etwa 200 Worte (etwa 20 Zeilen). Es sind keine Bilder zu verwenden.

\chapter*{Abstract}\addcontentsline{toc}{chapter}{Abstract}
ÖV-Güteklassen werden für die Beurteilung der Erschliessung mit dem öffentlichen Verkehr verwendet.
Die heute anerkannte Spezifikation der ÖV-Güteklassen des Bundesamts für Raumentwicklung (ARE)  basiert auf einer inzwischen ersetzten Schweizer Norm aus dem Jahre 1993.
Diverse Kantone haben in Eigeninitiativen Anpassungen daran vorgenommen, um kantonalen Gegebenheiten gerecht zu werden.
Die Implementationen dieser sind überholt und werden den aktuellen technischen Möglichkeiten nicht gerecht.
So führen einige Kantone die Berechnung dieser in Handarbeit durch.
Ebenfalls wird das Einzugsgebiet mit Luftlinien berechnet und die Topografie nicht konsequent in die Berechnung mit einbezogen.
Die kantonalen Eigenlösungen zeigen, dass noch kein Konsens einer aktuellen Lösung vorhanden ist.

Die neu erarbeiteten ÖV-Güteklassen 2018 kombinieren die Erkenntnisse einiger kantonalen Lösungen und verbessert diese mit den aktuellen technischen Möglichkeiten mit dem Ziel, eine allgemeingültige Spezifikation für die Schweiz zu erstellen.
In der Spezifikation wird nun vorausgesetzt, dass das Einzugsgebiet auf einem Fussgänger-Routing-Graphen berechnet wird.
Ebenfalls soll die Topografie konsequent berücksichtigt werden.
Diese Anforderung wird mit der Berechnung von Leistungskilometern erreicht.

Der ÖV-Güteklassen 2018 Generator setzt die neu erstellte Spezifikation um.
Er erzeugt einen schweizweiten Geodatensatz.
So werden mithilfe von OpenStreetMap-Daten und mit der pgRouting-Software (PostgreSQL) auf einem Fussgänger-Graphen für jedes Einzugsgebiet Isochronen berechnet.
Dadurch erhält man ein adäquates Verständnis über die Erreichbarkeit einer Haltestelle.
Durch den Einsatz des hochaufgelösten digitalen Terrainmodells swissALTI$^{3D}$ von Swisstopo wird die Topografie in die Berechnung mit einbezogen.
Eine Haltestelle auf einem schwer begehbaren Gelände hat somit ein kleineres Einzugsgebiet.
Die Berechnung des Kursintervalls einer Haltestelle wurde überdacht und eine neue Formel definiert.
Der Generator erlaubt die automatisierte Berechnung der ÖV-Güteklassen für verschiedene Stichtage.

Zur Veranschaulichung werden die berechneten ÖV-Güteklassen 2018 in einer
Webapplikation dargestellt.
Dabei können diese den ÖV-Güteklassen des ARE überlagert werden.

ÖV-Güteklassen 2018 geht mit der Zeit und passt sich denn aktuellen technischen Möglichkeiten an.
Durch die Konsolidierung einiger kantonaler Lösungen und Berücksichtigung der genauen Wegführung und Topografie wurde der erste Schritt in Richtung einer schweizweit anerkannter ÖV-Güteklassen-Spezifikation gemacht.

\cleardoublepage

\chapter*{Abstract}

Public transport quality gradings ("`ÖV-Güteklassen"') are used to measure the coverage radius of public transportation stations by distance and time.
The currently accepted specification by the Swiss Federal Office for Spatial Development (ARE) is based on an outdated standard from 1993.
Multiple cantons have since made adjustments to the methods to cater to their specific needs.
The varied alterations show that there is a need to create a general solution for Switzerland.
The cantonal adjustments and the current specification use linear distance for coverage radii and do not incorporate topographic data into the calculation.

This thesis introduces a new uniform specification for public transport quality gradings ("ÖV-Güteklassen 2018") which incorporates the varied approaches from the cantons and combines them with more modern technical capabilities.
The specification requires the coverage areas to be calculated along a pedestrian routing graph that incorporates topography of the area as well as the frequency of service.

Additionally, we created a software that implements the newly designed specification.
With the help of the routing engine "`pgRouting"', we create coverage areas along a pedestrian routing graph that allows for a more detailed understanding of the public transport coverage radius.
A high-resolution digital height model from swisstopo integrates the topography into this radius allowing a public transport stop on a steep hill to show a smaller area of coverage.
Furthermore, we define a new formula to calculate the time intervals of public transport stations and incorporate that into our quality measurement.

In order to evaluate the effectiveness  of the calculation for the new specification, we created a web application which allows the new coverage areas to be examined and compared with the current public transport gradings from ARE.
This web application allows for analysis of the differences between the two specifications and emphasizes the greater accuracy of the proposed new specification to form a unified public transit quality grading throughout the country.