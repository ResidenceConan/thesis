% Der Abstract richtet sich an den Spezialisten auf dem entsprechenden Gebiet
% und beschreibt daher in erster Linie die (neuen, eigenen) Ergebnisse und
% Resultate der Arbeit. Es umfasst nie mehr als eine Seite, typisch sogar nur
% etwa 200 Worte (etwa 20 Zeilen). Es sind keine Bilder zu verwenden.

\chapter*{Abstract}\addcontentsline{toc}{chapter}{Abstract}
ÖV-Güteklassen werden für die Beurteilung der Erschliessung mit dem öffentlichen Verkehr verwendet.
Die heute anerkannte Spezifikation der ÖV-Güteklassen des Bundesamts für Raumentwicklung (ARE)  basiert auf einer inzwischen ersetzten Schweizer Norm aus dem Jahre 1993.
Diverse Kantone haben in Eigeninitiativen Anpassungen daran vorgenommen, um kantonalen Gegebenheiten gerecht zu werden.
Die Implementationen dieser sind überholt und werden den aktuellen technischen Möglichkeiten nicht gerecht.
So führen einige Kantone die Berechnung dieser in Handarbeit durch.
Ebenfalls wird das Einzugsgebiet mit Luftlinien berechnet und die Topografie nicht konsequent in die Berechnung mit einbezogen.
Die kantonalen Eigenlösungen zeigen, dass kein Konsens in der aktuellen Lösung vorhanden ist.

Die Spezifikation ÖV-Güteklassen 2018 kombiniert die Erkenntnisse der kantonalen Lösungen und verbessert diese mit den aktuellen technischen Möglichkeiten mit dem Ziel, eine allgemeingültige Spezifikation für die Schweiz zu erstellen.
In der Spezifikation wird nun vorausgesetzt, dass das Einzugsgebiet auf einem Fussgänger-Routing-Graphen berechnet wird.
Ebenfalls soll die Topografie konsequent berücksichtigt werden.
Diese Anforderung wird mit der Berechnung von Leistungskilometern erreicht.

Der ÖV-Güteklassen 2018 Generator setzt die neu erstellte Spezifikation um.
So werden für das Einzugsgebiet mithilfe von pgRouting auf einem Fussgänger-Routing-Graphen Isochronen berechnet.
Dadurch erhält man ein detaillierteres Verständnis über die Erreichbarkeit der Haltestelle.
Durch den Einsatz eines hoch aufgelösten digitalen Terrainmodells von swisstopo wird die Topografie in die Berechnung miteinbezogen. Eine Haltestelle auf einem schwer begehbaren Gelände hat somit ein kleineres Einzugsgebiet.
Die Berechnung des Kursintervalls einer Haltestelle wurde überdacht und eine neue Formel definiert.
Der Generator erlaubt die automatisierte Berechnung der ÖV-Güteklassen für verschiedene Stichtage.

Zur Auswertung werden die aufgrund der neuen Spezifikation berechneten ÖV-Güteklassen in einer Webapplikation dargestellt.
Dabei können diese den ÖV-Güteklassen des Bundesamts für Raumentwicklung visuell überlagert werden.


\cleardoublepage

\chapter*{Abstract}

%TODO