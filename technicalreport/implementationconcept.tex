% z.T. Wiederholung im Groben, z.T. Verweise auf Teil II-Kapitel

\section{Umsetzungskonzept}
\label{Umsetzungskonzept}

In einer ersten Phase wurde die \acs{ÖV}-Güteklassen 2018 Spezifikation in Kapitel \ref{Spezifikation OeVGK18} erstellt.
Dabei wurde die Spezifikation so gestaltet, dass diese den aktuellen technischen Möglichkeiten gerecht wird.
Dies impliziert unter anderem die Verwendung eines Routing-Graphen für die Berechnung des Einzuggebiets und das Einbeziehen eines hoch aufgelösten digitalen Terrainmodells.
Dabei werden zusätzliche Fehler in den bisherigen Lösungen ausgemerzt.

In einer zweiten Phasen wird inkrementell die Spezifikation umgesetzt.
Dabei wird ein Generator entwickelt, welcher die \acs{ÖV}-Güteklassen aufgrund der in der ersten Phase erarbeiteten Spezifikation berechnet.
Parallel dazu wird eine Web-Applikation mit zugehörigem Backend erarbeitet, welche die berechneten \acs{ÖV}-Güteklassen 2018 darstellt und visuell dem Platzhirsch (Spezifikation des \acl{ARE}) gegenüberstellt.
Diese Web-Applikation verfolgt das Ziel, Verkehrs-, Raumplaner sowie Privatpersonen einen Einblick in die Qualität der ÖV-Erschliessung an einem Standort geben zu können.
Die zweite Phase ist im Teil \ref{Technischer Bericht} beschrieben.
