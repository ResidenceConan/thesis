% z.T. Wiederholung im Groben, z.T. Verweise auf Teil II-Kapitel

\section{Umsetzungskonzept}
\label{Umsetzungskonzept}

Die Arbeit wird in zwei von einander abhängigen Phasen umgesetzt.
Die erste Phase ist theoretisch gestaltet.
Dieser wird grosse Bedeutung beigemessen, da die zweite Phase auf den Ergebnissen dieser aufbaut.
So verfolgt die erste Phase das Ziel eine \acs{ÖV}-Güteklassen 2018 Spezifikation zu erstellen, mit dem Fokus schweizweit Akzeptanz zu erhalten.
Dies hat die Implikation, dass bestehende Lösungen, sprich die Spezifikation des \acl{ARE} und verschiedenen ausgewählten kantonale Ausführungen analysiert werden müssen, um so einen Konsens erreichen zu können.
Bewährtes und Akzeptiertes wird kritisch hinterfragt, aber es wird nicht mit Biegen und Brechen eine neuer Weg eingeschlagen.
Dabei wird die Spezifikation so gestaltet, dass diese den aktuellen technischen Möglichkeiten gerecht wird.
Dies impliziert unter anderem die Verwendung eines Routing-Graphen für die Berechnung des Einzuggebiets und das Einbeziehen eines hoch aufgelösten digitalen Terrainmodells.
Dabei werden zusätzliche Fehler in den bisherigen Lösungen ausgemerzt.

In einer zweiten Phasen wird inkrementell die Spezifikation umgesetzt.
Dabei wird ein Generator entwickelt, welcher die \acs{ÖV}-Güteklassen aufgrund der in der ersten Phase erarbeiteten Spezifikation berechnet.
Parallel dazu wird eine Webapplikation mit zugehörigem Backend erarbeitet, welche die berechneten \acs{ÖV}-Güteklassen 2018 darstellt und visuell dem Platzhirsch (Spezifikation des \acl{ARE}) gegenüberstellt.
Diese Webapplikation verfolgt das Ziel, Verkehrs-, Raumplaner sowie Privatpersonen einen Einblick in die Qualität der ÖV-Erschliessung an einem Standort geben zu können.
