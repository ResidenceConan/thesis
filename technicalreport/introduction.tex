
\section{Einführung}
\label{Einführung}

%TODO

\subsection{Grundlagen und Begriffe}
\label{Einführung:Grundlagen und Begriffe}

Folgend werden Grundlagen und Begriffe eingeführt, welche für das grundlegende Verständnis der Arbeit relevant sind.

\subsubsection{ÖV-Güteklasse}
\label{Grundlagen und Begriffe:ÖV-Güteklasse}

\acs{ÖV}-Güteklassen geben Auskunft darüber, wie gut ein Standort mit dem öffentlichen Verkehr erschlossen ist.
Dies ist wichtig, wenn es darum geht, den öffentlichen Verkehr zu optimieren, Siedlungsverdichtung nach Innen an geeigneten Lagen voranzutreiben oder Standortentscheide für publikumsintensive Anlagen so zu treffen, dass sie möglichst wenig zusätzlichen Autoverkehr verursachen.~\cite{oev-guteklasse-gr-defintion}


\subsection{Problemstellung}
\label{Einführung:Problemstellung}

Mit \acs{ÖV}-Güteklassen wird ermittelt, welche Gebiete wie gut mit öffentlichen Verkehrsmitteln erschlossen sind.
Die Daten werden unter anderem von Raumplanern und Verkehrsplanern verwendet, um Entscheidungen für die Entwicklung von Gebieten zu treffen.

Die Methodik zur Erhebung von Güteklassen wurde 1993 erstmals in der Schweizer Norm SN 640 290~\cite{sn640290} festgelegt.
Die Norm wurde damals zur Berechnung des Bedarfs an Parkplätzen eingeführt.
Sie wurde 2006 durch die neue Norm SN 640 281~\cite{sn640281} ersetzt.
Darin sind aber Güteklassen in dieser Form nicht mehr erhalten. Die alte Norm wird aber immer noch als Grundlage verwendet, um die aktuellen \acs{ÖV}-Güteklassen zu berechnen~\cite{berechnung_are}.

Die Berechnungsmethoden, wie sie in der Norm festgehalten wurden, sind für die heutigen technischen Möglichkeiten überholt.
Im Folgenden werden einige Probleme und Lösungsansätze für die Berechnung von Güteklassen im Bezug auf öffentliche Verkehrsmittel aufgezeigt.

\subsubsection{Luftlinie bei Einzugsgebieten}
\label{problem:Luftlinie bei Einzugsgebieten}

Bei der Berechnung von Güteklassen nach der Norm SN 640 290~\cite{sn640290} wird um jede Haltestelle Kreise mit einem gewissen Radius gezogen.
Diese bestimmen die Einzugsgebiete der Haltestellen mit unterschiedlicher Qualität.
Diese Methode beachtet aber nicht die effektive Erreichbarkeit für Fussgänger.
Es kann z.B. sein, dass die Haltestelle nur durch eine Strasse zugänglich ist und Fussgänger in der Umgebung einen Umweg laufen müssen.
Das effektive Einzugsgebiet wird so also nicht akkurat repräsentiert.

\subsubsection{Topografie}
\label{problem:Topografie}

In der jetzigen Berechnung der Güteklassen wird die Topografie der Umgebung von Haltestellen nicht einberechnet.
Je steiler der Zugang zu einer Haltestelle ist, desto kleiner wird das effektive Einzugsgebiet der Haltestelle, da Fussgänger weniger bereit sind, Distanzen zu laufen, die auf ebenem Grund akzeptabel wären.

\subsubsection{Taktberechnung}
\label{problem:Taktberechnung}

Nach der verwendeten Norm SN 640 290~\cite{sn640290} werden Haltestellen anhand der Anzahl der Verbindungen in einer bestimmten Zeit in Kategorien eingeteilt.
Je höher diese Kategorie, desto grösser wird das Einzugsgebiet der Haltestelle bewertet.

Bei der Bewertung der Kategorie wird aber nicht darauf geachtet, in welchem Takt die Verbindungen eintreffen.
So kann es sein, dass an einer Haltestelle alle 15 Minuten genau ein Bus ankommt, bei einer anderen alle 30 Minuten gleich zwei Busse in gleicher Richtung hintereinander.
Die Berechnung zählt aber nur die Anzahl Verbindungen insgesamt, womit beide Haltestellen danach gleichwertig wären, obwohl die Haltestelle mit dem 15 Minuten Takt offensichtlich besser angeschlossen ist.

In der Kategorisierung der Norm werden ausserdem Haltestellen, die halbstündlich oder stündlich bedient werden, in die gleiche Kategorie eingeteilt.
Für ein Benützer der Haltestelle ist dies allerdings ein markanter Unterschied.

\subsubsection{Berechnung an einem Stichtag}
\label{problem:Berechnung an einem Stichtag}

Bei der Berechnung der aktuellen Güteklassen vom \ac{ARE}~\cite{berechnung_are} werden für die Kategorisierung von Haltestellen die Verbindungen an einem gewissen Stichtag --- ein Werktag ausserhalb der touristischen Hochsaison --- definiert.
Dabei werden nur Verbindungen zwischen 6.00 Uhr und 20.00 Uhr gezählt.
Die Ansprüche an die Güteklassen können aber auch andere Zeiten umfassen.
So kann es interessant sein, die Verbindungen in der Nacht oder am Wochenende einzubeziehen.
Für touristische Regionen ist die Situation während der Hochsaison vielleicht interessanter als an einem normalen Werktag.

\subsection{Vision}
\label{Einführung:Vision}

\acs{ÖV}-Güteklassen 2018 ermöglicht es unterschiedlichen Stakeholdern die Qualität der öffentlichen Erschliessung eines Standortes unter Berücksichtigung des Höhenmodells und der Streckenführung in Kombination mit Verbesserungen aus den Erfahrungen mit dem alten Modell zu analysieren, um so Entscheidungen, welche den öffentlichen Raum betreffen, fundiert treffen zu können.

\subsection{Ziele und Unterziele}
\label{Einführung:Ziele und Unterziele}

%TODO

\subsubsection{Prototyp und Deliverables}
\label{Ziele und Unterziele:Prototyp und Deliverables}

Das Resultat der Arbeit besteht aus zwei Teilen.
Für den ersten Teil wird die Spezifikation ''\acs{ÖV}-Güteklassen 2018'' definiert, die auf der Norm SN 640 290~\cite{sn640290} sowie der aktuellen Berechnungsmethodik des \ac{ARE}~\cite{berechnung_are} aufsetzt und diese mit den oben genannten Punkten erweitert.

Im zweiten Teil wird die erarbeitete Spezifikation umgesetzt.
Mit einem reproduzierbarem Skript werden die \acs{ÖV}-Güteklassen für mehrere Parameter berechnet.
% TODO wo werden Parameter definiert?
In einem weiteren Schritt wird das Ergebnis der Berechnungen in einer Web-Applikation visualisiert.
Darin können neben unseren Berechnungen auch die bisherigen \acs{ÖV}-Güteklassen (''\acs{ÖV}-Güteklassen 1993'') angezeigt werden.

\subsection{Rahmenbedingungen, Umfeld, Definitionen und Abgrenzungen}
\label{Einführung:Rahmenbedingungen, Umfeld, Definitionen, Abgrenzungen}

%TODO

\subsection{Vorgehen und Aufbau der Arbeit}
\label{Einführung:Vorgehen und Aufbau der Arbeit}

Die Bachelorarbeit befasst sich mit dem Problem, eine moderne Erweiterung der ''\acs{ÖV}-Güteklassen 1993'' zu entwickeln und die Berechnung umzusetzen.
Im ersten Schritt werden die aktuellen Berechnungsmethoden analysiert und ermittelt, welche Anpassungen und Erweiterungen in die neue Spezifikation ''\acs{ÖV}-Güteklassen 2018'' übernommen werden.
Ausserdem wird geprüft, inwiefern und mit welchen Technologien die Berechnung möglichst effizient durchgeführt und visualisiert werden kann.

In einem zweiten Teil wird die erarbeitete Spezifikation implementiert.
Mit aktuellen Kartendaten von \ac{OSM} und den Fahrplandaten für das Jahr 2018 werden die \acs{ÖV}-Güteklassen für die gesamte Schweiz berechnet.
Die Visualisierung erfolgt in Form einer Web-Applikation, worin die \acs{ÖV}-Güteklassen auf einer interaktiven Karte dargestellt sind.

Die Erarbeitung sowie das Ergebnis der Spezifikation ''\acs{ÖV}-Güteklassen 2018'' wird im Teil \ref{Technischer Bericht} beschrieben.
Die Berechnung und Visualisierung der Spezifikation wird im Teil \ref{SW-Projektdokumentation} behandelt.
