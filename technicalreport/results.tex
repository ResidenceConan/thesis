
\section{Resultate}
\label{Resultate}

In diesem Kapitel werden die Resultate der Arbeit präsentiert und mit den im Kapitel \ref{Stand der Technik:Aktuelle Situation} vorgestellten bestehenden Berechnungsmethoden verglichen.
Die technische Beschreibung zur Implementation befindet sich im Teil \ref{SW-Projektdokumentation} Kapitel \ref{Implementation}.

\subsection{Zielerreichung}
\label{Resultate:Zielerreichung}

In der Evaluations-Phase (Kap. \ref{Stand der Technik}) wurden bestehende Ansätze zur Berechnung von \acs{ÖV}-Güteklassen analysiert und verglichen.
Es zeigte sich, dass die Methoden auf einer alten Norm aufbauen, die von einigen Kantonen etwas erweitert wurden.
Für die Berechnung der ganzen Schweiz gibt es aber keine Spezifikation, die moderne Analysen wie die Berücksichtigung des Strassennetzes vorsehen.

Unter Zuhilfenahme der Berechnungsmethodik des Kanton Graubündens erstellten wir anschliessend eine eigene Spezifikation, die "`\acs{ÖV}-Güteklassen 2018"'.
Dabei haben wir neben einer genaueren Ermittlung der Erreichbarkeit (Kap. \ref{Verbesserungsmöglichkeiten:Ermittlung der Erreichbarkeit der Haltestelle}) und die Berücksichtigung des Terrains auch Punkte konkretisiert, die in bisherigen Methoden nicht genau spezifiziert sind, wie etwa die Berechnung des Kursintervalls (Kap. \ref{Verbesserungsmöglichkeiten:Kursintervall}) oder die Bestimmung von Bahnknoten (Kap. \ref{Verbesserungsmöglichkeiten:Bestimmung der Bahnknoten}).

In einem nächsten Schritt wurde die Berechnung anhand der "`\acs{ÖV}-Güteklassen 2018"'-Spezifikation umgesetzt und automatisiert.
So lassen sich die \acs{ÖV}-Güteklassen auch für nachfolgende Jahre neu berechnen.
Alle Parameter, die in der Spezifikation festgelegt wurden, sind in einer Konfigurationsdatei festgehalten und können beliebig verändert werden.
Dadurch kann die Implementation auch als Basis für Weiterentwicklungen der Berechnungsmethoden dienen.

\begin{figure}[ht]
    \centering
    \includegraphics[width=1\linewidth]{technicalreport/img/resultat_oevgk18_uebersicht}
    \caption[Darstellung der berechneten ÖV-Güteklassen in der Web-Applikation]{Darstellung der berechneten ÖV-Güteklassen in der Web-Applikation}
    \label{fig:resultat_webapp_uebersicht}
\end{figure}

Zusätzlich zur automatisierten Berechnung wurde eine Web-Applikation entwickelt, die es ermöglicht, die berechneten \acs{ÖV}-Güteklassen im Browser darzustellen (siehe Abbildung \ref{fig:resultat_webapp_uebersicht}).
Ebenfalls wurden darin die bisherigen \acs{ÖV}-Güteklassen vom Bundesamt für Raumentwicklung (\acs{ARE}) integriert, was einen visuellen Vergleich der beiden Resultate erleichtert.

\subsection{Vergleich mit bisherigen ÖV-Güteklassen}
\label{Resultate:Vergleich mit bisherigen ÖV-Güteklassen}

Im Folgenden wird an einigen Beispielen gezeigt, wie die Resultate unserer Berechnung der "`\acs{ÖV}-Güteklassen 2018"' von bisherigen Methoden abweichen.
Als Vergleich werden dazu die \acs{ÖV}-Güteklassen von \ac{ARE} aus dem Jahre 2018~\cite{berechnung_are} verwendet.

\subsubsection{Einfluss der Wegführung}

\begin{figure}[ht]
    \centering
    \includegraphics[width=0.8\linewidth]{technicalreport/img/vergleich_wegfuehrung}
    \caption[Vergleich zur Einfluss der Wegführung]{Vergleich zur Einfluss der Wegführung. Oben: ÖV-Güteklassen von \ac{ARE}. Unten: ÖV-Güteklassen 2018.}
    \label{fig:vergleich_wegfuehrung}
\end{figure}

Wie in \ref{Problemstellung:Luftlinie bei Einzugsgebieten} besprochen wird bei den bisherigen Ansätzen das Einzugsgebiet einer Haltestelle mit einem Kreis, und somit als Luftlinie, beschrieben.
Mit "`\acs{ÖV}-Güteklassen 2018"' wird nun das Einzugsgebiet durch eine \gls{Isochrone} beschrieben.
Die Fläche zeigt das Einzugsgebiet, das von einem Fussgänger von der Haltestelle aus in der definierten Zeit erreichbar ist.

Wie in Abbildung \ref{fig:vergleich_wegfuehrung} zu sehen ist, entstehen dadurch Geometrien, die dem Strassen- und Wegnetz folgen.
Dies ergibt einen realistischeren Eindruck über die effektiven Gebiete, die von einer Haltestelle erschlossen werden.

\subsubsection{Verbesserte Berechnung des Kursintervalls}

TODO

\subsubsection{Verbesserte Ermittlung von Bahnknoten}

TODO

\subsection{Ausblick: Weiterentwicklung}
\label{Resultate:Ausblick: Weiterentwicklung}

Durch die neue Spezifikation und Web-Applikation ist die Grundlage für weitere und fortgeschrittene Auswertungen geschaffen.

Verknüpft man die neuen \acs{ÖV}-Güteklassen mit der Bevölkerungsdichte, ist es möglich, Verbesserungspotential für den \acs{ÖV} zu eruieren.
Dadurch wären Auswertungen analog zu "`95\% der wohnhaften Personen in einem bestimmten Gebiet müssen mindestens in der \acs{ÖV}-Güteklassen XYZ sein"' möglich.

Aber auch für den privaten Gebrauch ist eine Weiterentwicklung von Interesse.
So ist ein Wohnortfinder denkbar, welcher die Qualität des \acs{ÖV}-Anschlusses und anderen Faktoren (Nähe zu einer Schule oder Krankenhaus, etc.) berücksichtigt.

Für die Berechnung der Einzugsgebiete wird in dieser Arbeit der Ansatz von \glspl{Isochrone} verwendet, die entlang des Strassen- und Wegnetzes verlaufen.
Die Grundannahme dahinter ist, dass sich Fussgänger immer entlang des Wegnetzes fortbewegen.
Dabei gibt es Probleme bei offenen Fussgängerflächen, wie sie in~\cite{plaza_route} behandelt werden.
Ein anderer Ansatz basiert statt auf Routen auf einem Raster~\cite{pedestrian_accessibility_planning}. Dieser geht davon aus, dass sich Fussgänger überall fortbewegen, wo keine Hindernisse (Gebäude, Flüsse, Barrieren, etc.) sie daran hindern.
Dabei wird ein Raster über die Karte gelegt und jeder Kachel einen Kostenwert zugeordnet.
Für die Erzeugung der Einzugsgebiete werden dann vom Startpunkt aus in alle Richtungen die Kacheln eingefärbt und bei jedem Schritt die Kosten der Kachel addiert.
Dies wird so lange wiederholt, bis die gewünschten Maximalkosten erreicht werden.
In Abbildung \ref{fig:grid_based_approach} ist ein Ergebnis einer solchen Berechnung abgebildet.

\begin{figure}[ht]
    \centering
    \includegraphics[width=0.6\linewidth]{start/img/grid_based_approach.png}
    \caption[Grid- beziehungsweise raster-basierter Ansatz]{Grid- beziehungsweise raster-basierter Ansatz~\cite{pedestrian_accessibility_planning}}
    \label{fig:grid_based_approach}
\end{figure}

Mit dieser Methode ist die Qualität des Routing-Graphen und Problematiken wie bei Fussgängerflächen weniger relevant.
Für die Berechnung von \acs{ÖV}-Güteklassen wäre dies ein spannender Ansatz für den Vergleich mit der Lösung in dieser Arbeit.



\subsection{Dank}
\label{Resultate:Dank}

Wir möchten folgenden Personen für ihre Unterstützung und Mitwirkung bei dieser Arbeit danken:

\textbf{Prof. Stefan Keller, IFS Institut für Software,} für die Zeit, Ressourcen, Kontakte, Know-How und Unterstützung, von welcher wir jederzeit profitieren konnten.

\textbf{Prof. Claudio Büchel, IRAP Institut für Raumentwicklung}, für die initiale Idee, Know-How über Verkehrsplanung und Fachunterstützung über das gesamte Projekt.

\textbf{Mitarbeiter, IFS Institut für Software,} für den regen Know-How-Austausch und die Unterstützung bei der Produktivsetzung.
