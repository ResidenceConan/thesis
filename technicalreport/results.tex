
\section{Resultate}
\label{Resultate}

%TODO

\subsection{Zielerreichung}
\label{Resultate:Zielerreichung}

%TODO

\subsection{Ausblick: Weiterentwicklung}
\label{Resultate:Ausblick: Weiterentwicklung}

%Durch die neue Spezifikation und Web-Applikation ist die Grundlage für weitere und fortgeschrittene Auswertungen geschaffen.
%Verknüpft man die neuen \acs{ÖV}-Güteklassen mit der Bevölkerungsdichte, ist es möglich automatisch Verbesserungspotential für den \acs{ÖV} zu eruieren.
%Dadurch wären Auswertungen analog zu \emph{"`95\% der wohnhaften Personen in einem bestimmten Gebiet müssen mindestens in der \acs{ÖV}-Güteklassen XYZ sein"'} möglich.

%Aber auch für den privaten Gebrauch ist eine Weiterentwicklung von Interesse.
%So ist ein Wohnortfinder, welcher die Qualität des \acs{ÖV}-Anschluss und anderen Faktoren (Nähe zu einer Schule oder Krankenhaus, etc.) berücksichtigt, eine Option, welche in Betracht gezogen werden kann.


\subsection{Dank}
\label{Resultate:Dank}

Wir möchten folgenden Personen für ihre Unterstützung und Mitwirkung bei dieser Arbeit danken:

\textbf{Prof. Stefan Keller, IFS Institut für Software,} für die Zeit, Ressourcen, Kontakte, Know-How und Unterstützung, von welcher wir jederzeit profitieren konnten.

\textbf{Prof. Claudio Büchel, IRAP Institut für Raumentwicklung}, für die initiale Idee, Know-How über Verkehrsplanung und Fachunterstützung über das gesamte Projekt.

\textbf{Mitarbeiter, IFS Institut für Software,} für den regen Know-How-Austausch und die Unterstützung bei der Produktivsetzung.
