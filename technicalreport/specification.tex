
\section{Spezifikation OeVGK18}
\label{Spezifikation OeVGK18}

\subsection{Zusammenhang zur Berechnungsmethodik ARE}
\label{Spezifikation OeVGK18:Zusammenhang zur Berechnungsmethodik ARE}
In der Weisung des \acs{UVEK} vom 16.02.2015 ist festgehalten, dass für die Beurteilung der Qualität der Erschliessung mit dem öffentlichen Verkehr die \nameref{Lösungsansätze:Berechnungsmethodik ARE} verwendet wird~\cite{weisung_uvek}.
Dies ist der Grund warum im Folgenden der Bogen zu dieser Berechnungsmethodik geschlagen wird und die vorgeschlagenen Änderungen dieser Grundlagen gegenüber gestellt wird.
Learnings und gemachte Anpassungen werden, wo es Sinn macht, übernommen und so gekennzeichnet. 

\subsection{Berechnungsmethodik OeVGK18}
\label{Berechnungsmethodik OeVGK18}
Folgend wird die Berechnungsmethodik \gls{OeVGK18} detailliert beschrieben.

\subparagraph{Art der Verkehrsmittel}~\\
\label{Berechnungsmethodik OeVGK18:Art der Verkehrsmittel}

Verkehrsmittel werden in folgende Verkehrsmittelgruppen eingeteilt:

\begin{itemize}[noitemsep]
    \item Verkehrsmittelgruppe A
    \begin{itemize}
        \item Bahnknoten
    \end{itemize}
    \item Verkehrsmittelgruppe B
    \begin{itemize}
        \item Bahnlinie
    \end{itemize}
    \item Verkehrsmittelgruppe C
    \begin{itemize}
        \item Tram, Bus, Postauto, Rufbus, Schiff, Seilbahn
    \end{itemize}
\end{itemize}

\paragraph{Bahnknoten}~\\
TODO: Definition Bahnknoten

\paragraph{Schiffe und Seilbahnen}~\\
Schiffe und Seilbahnen werden nur berücksichtigt, wenn sie in \emph{TODO} und \emph{TODO}.
% Falls wir den Weg mit dem 2. Dienstag im November wählen, ist diese Definition überflüssig


\subparagraph{Kursintervall}~\\
\label{Berechnungsmethodik OeVGK18:Kursintervall}
Es sind 2 Stichtage mit jeweils zwei Zeitbereichen zu definieren, welche ausserhalb der Ferienzeit und der touristischen Hochsaison liegen.

\begin{longtable}[c]{l l}
    \midrule
    \textbf{Stichtag}
                            & \textbf{Zeitbereich}\\
    Werktag
                            & 06.00 - 20.00 Uhr\\
    Werktag
                            & 22.00 - 05.00 Uhr\\
    Samstag
                            & 06.00 - 20.00 Uhr\\
    Samstag
                            & 22.00 - 05.00 Uhr\\
    \bottomrule
\caption{Kursintervall: Stichtag und Zeitbereich}
\label{table:Ermittlung Kursintervall: Stichtag und Zeitbereich}
\end{longtable}

TODO Kursberechnung

\subparagraph{Haltestellenkategorie}~\\
\label{Berechnungsmethodik OeVGK18:Haltestellenkategorie}
%TODO Transportmittelkategorie vs Verkehrsmittelkategorie
Die Haltestellenkategorie I bis VII wird mit folgender Tabelle eruiert:

\begin{longtable}[c]{l p{4.0cm} p{4.0cm} p{4.0cm}}
    \midrule
    \textbf{}
                            & \multicolumn{3}{l}{\textbf{Verkehrsmittelgruppe}}\\
    \textbf{Kursintervall}
                            & \textit{A}
                            & \textit{B}
                            & \textit{C}\\
    \textit{< 5 min}
                            & I
                            & I
                            & II\\
    \textit{5 - 10 min}
                            & I
                            & II
                            & III\\
    \textit{11 - 19 min}
                            & II
                            & III
                            & IV\\
    \textit{20 - 39 min}
                            & III
                            & IV
                            & V\\
    \textit{40 - 60 min}
                            & IV
                            & V
                            & VI\\
    \textit{> 60 min}
                            &
                            & VII
                            & VII\\
    \bottomrule
\caption{Haltestellenkategorie}
\label{Haltestellenkategorie}
\end{longtable}

\subparagraph{Gehzeit zur Haltestelle}~\\
\label{Berechnungsmethodik OeVGK18:Distanz zur Haltestelle}
Bei der Berechnung der Gehzeit zu einer Haltestelle ist eine Laufgeschwindigkeit von $1.4 m/s$ anzunehmen und die Strecke entlang des Wege- und Strassennetzes zu berücksichtigen, welche im folgenden als Horizontaldistanz bezeichnet wird.

Damit man der Topografie gerecht wird, ist als massgebende zu laufende Distanz Leistungsmeter zu verwenden:
\[
    x = 
\begin{cases}
    a + b/0.1 + c/0.15, & \text{if } c/d> 0.22\\
    a + b/0.1,          & \text{otherwise}
\end{cases}
\]
\begin{conditions}
    x   &   Leistungsmeter [m]\\
    a   &   Horizontaldistanz [m]\\
    b   &   positive Steigung in Höhenmeter [m]\\
    c   &   negative Steigung in Höhenmeter [m]\\
    d   &   Horizontaldistanz mit negativer Steigung [m]
\end{conditions}

Die Gehzeit $t$ zur Haltestelle ergibt sich nun aus:
\[ t = \frac{x}{1.4 m/s} \]


\subparagraph{ÖV-Güteklassen}~\\
\label{Berechnungsmethodik OeVGK18:ÖV-Güteklassen}
Die Kombination aus Haltestellenkategorie und Distanz zur Haltestelle liefert folgende \acs{ÖV}-Güteklassen-Gruppierung:

\begin{longtable}[c]{l p{3.3cm} p{3.3cm} p{3.3cm} p{3.3cm}}
    \midrule
    \textbf{}
                            & \textbf{< 300 s}
                            & \textbf{301 - 450 s}
                            & \textbf{451 - 600 s}
                            & \textbf{601 - 900 s}\\
    \textbf{I}
                            & Klasse A
                            & Klasse A
                            & Klasse B
                            & Klasse C\\
    \textbf{II}
                            & Klasse A
                            & Klasse B
                            & Klasse C
                            & Klasse D\\
    \textbf{III}
                            & Klasse B
                            & Klasse C
                            & Klasse D
                            &\\
    \textbf{IV}
                            & Klasse C
                            & Klasse D
                            &
                            &\\
    \textbf{V}
                            & Klasse D
                            &
                            &
                            &\\
    \textbf{VI}
                            & Klasse E
                            &
                            &
                            &\\
    \textbf{VII}
                            & Klasse F
                            &
                            &
                            &\\                                
    \bottomrule
\caption{ÖV-Güteklassen}
\label{table:ÖV-Güteklassen}
\end{longtable}
