
\section{Stand der Technik}
\label{Stand der Technik}

\subsection{Aktuelle Situation}
\label{Stand der Technik:Aktuelle Situation}

\subsubsection{Schweizer Norm 640 290}
\label{Aktuelle Situation:Schweizer Norm 640 290}

\acs{ÖV}-Güteklassen wurden erstmals mit der \ac{SN} 640 290~\cite{sn640290} des Vereins Schweizerischer Strassenfachleute (VSS) im Jahre 1993 für die Schweiz definiert.
Diese Norm enthält Richtwerte für die Bestimmung des Grenzbedarfs an Parkfeldern für Personenwagen und führte hierfür die Definition der \acs{ÖV}-Güteklassen ein.
Im Jahre 2006 wurde diese durch die \acs{SN} 640 281 abgelöst, welche nicht mehr auf die \acs{ÖV}-Güteklassen eingeht.
Es wurde seither keine allgemeingültige Definition verabschiedet.

Im Folgenden ist aufgeschlüsselt, was genau die \acs{SN} 640 290~\cite{sn640290} in Bezug auf \acs{ÖV}-Güteklassen definiert und wo Schwachstellen liegen.

\paragraph{Definition ÖV-Güteklassen}~\\
\label{Schweizer Norm 640 290:Definition ÖV-Güteklassen}
Grundsätzlich definieren \acs{ÖV}-Güteklassen die Qualität der Erschliessung durch den öffentlichen Verkehr.
Diese Qualität wird durch die \nameref{Definition ÖV-Güteklassen:Bedienungsqualität der Haltestelle} und die \nameref{Definition ÖV-Güteklassen:Erreichbarkeit der Haltestelle} festgesetzt.
Beide Kriterien sind folgend definiert.
Anmerken muss man, dass zu besserer Verständlichkeit und Übersichtlichkeit unter anderem zusätzliche Terminologie wie Transportmittelkategorie eingeführt wird.

\subparagraph{Bedienungsqualität der Haltestelle}~\\
\label{Definition ÖV-Güteklassen:Bedienungsqualität der Haltestelle}
Die öffentlichen Verkehrsmittel werden zuerst in die in Tabelle \ref{table:Transportmittelkategorien} ersichtlichen Transportmittelkategorien eingeteilt.

\begin{longtable}{l p{10.6cm}}
        \midrule
        \textbf{Transportmittelkategorie 1}
                                & Bahnknoten: mehrere Bahnlinien in unterschiedliche Richtungen\\
        \textbf{Transportmittelkategorie 2}
                                & Bahnlinie\\
        \textbf{Transportmittelkategorie 3}
                                & Tramway, Trolleybuy, Autobus, Regionalbus, Ortsbus mit gutem Anschluss an Bahnlinie\\
        \textbf{Transportmittelkategorie 4}
                                & Ortsbus, lokaler Kleinbus\\
        \bottomrule
    \caption{Transportmittelkategorien}
    \label{table:Transportmittelkategorien}
\end{longtable}

Die Transportmittel werden dabei in die Gruppe A und B zusammengefasst, welche für die Berechnung des durchschnittlichen Kursintervalls berücksichtigt werden.
Dabei werden für die Intervallberechnung alle Verkehrsmittel der Transportmittelkategorie in einer Gruppe berücksichtigt.

\begin{longtable}[c]{l | p{2.3cm} p{2.3cm} | p{2.3cm} p{2.3cm}}
        \midrule
        \textbf{}
                                & \multicolumn{2}{l|}{\textbf{Gruppe A}}
                                & \multicolumn{2}{l}{\textbf{Gruppe B}}\\
        \textbf{Transportmittelkategorie}
                                & \textit{1}
                                & \textit{2}
                                & \textit{3}
                                & \textit{4}\\
        \textbf{Kursintervall}
                                &
                                &
                                &
                                &\\
        \textit{< 5 min}
                                & I
                                & I
                                & II
                                & III\\
        \textit{5 - 9 min}
                                & I
                                & II
                                & III
                                & IV\\
        \textit{10 - 19 min}
                                & II
                                & III
                                & IV
                                & V\\
        \textit{20 - 39 min}
                                & III
                                & IV
                                & V
                                & V\\
        \textit{40 - 60 min}
                                & IV
                                & V
                                & V
                                &\\
        \bottomrule
    \caption{Ermittlung der Haltestellenkategorie}
    \label{table:Ermittlung der Haltestellenkategorie}
\end{longtable}

Der Kursintervall ist der durchschnittliche Abstand zwischen Ankunft beziehungsweise Abfahrt (pro Linie jeweils in der Hauptlastrichtung) aller Verkehrsmittel in der gleichen Gruppe zwischen 06:00 - 20:00 Uhr jeweils von Montag bis Freitag.
Die Norm macht hier Ausnahmen bei Verdichtungen in Hauptverkehrszeiten, Linienüberlagerungen und reinen Arbeitsplatzgebieten mit stark verdichtetem Fahrplan während Pendlerzeiten.

\subparagraph{Erreichbarkeit der Haltestelle}~\\
\label{Definition ÖV-Güteklassen:Erreichbarkeit der Haltestelle}
Basierend auf den vorhin definierten Haltestellenkategorien (I bis V) kann die Erreichbarkeit pro Haltestellenkategorien definiert werden.
Die Distanzen sind Luftlinien, wobei ein mittlerer Umwegfaktor von 20 bis 30\% berücksichtigt wird.

\begin{longtable}[c]{l p{3.3cm} p{3.3cm} p{3.3cm} p{3.3cm}}
        \midrule
        \textbf{}
                                & \textbf{< 300 m}
                                & \textbf{300 - 500 m}
                                & \textbf{501 - 750 m}
                                & \textbf{751 - 1000m}\\
        \textbf{I}
                                & Klasse A
                                & Klasse A
                                & Klasse B
                                & Klasse C\\
        \textbf{II}
                                & Klasse A
                                & Klasse B
                                & Klasse C
                                & Klasse D\\
        \textbf{III}
                                & Klasse B
                                & Klasse C
                                & Klasse D
                                &\\
        \textbf{IV}
                                & Klasse C
                                & Klasse D
                                &
                                &\\
        \textbf{V}
                                & Klasse D
                                &
                                &
                                &\\
        \bottomrule
    \caption{Ermittlung der Erreichbarkeit der Haltestelle}
    \label{table:Ermittlung Erreichbarkeit der Haltestelle}
\end{longtable}

Der Topographie wird Rechnung getragen, indem man bei schwierigen Gegebenheiten die nächste Klasse wählt oder die Distanz vergrössert.

\subsection{Lösungsansätze}
\label{Stand der Technik:Lösungsansätze}
Basierend auf der bestehenden \nameref{Aktuelle Situation:Schweizer Norm 640 290} wurde im Jahre 2011 die Berechnungsmethodik des \ac{ARE} veröffentlicht, welche in einer Weisung des \acs{UVEK} vom 16.02.2015 als die Berechnugsmethodik definiert wird, welche bei der Beurteilung der Qualität der Erschliessung mit dem öffentlichen Verkehr verwendet werden soll.

In der Zwischenzeit wurden einige kantonale Anpassungen an dieser Methodik und der \nameref{Aktuelle Situation:Schweizer Norm 640 290} vorgenommen.
Folgend werden grob die Abweichungen der Norm der verabschiedeten Berechnungsmethodik ARE und zweier kantonaler Methodiken beschrieben, namentlich des Kanton Graubünden und Zürich.

Im Anschluss werden die Methodiken zusammen betrachtet, Schwachstellen eruiert und Verbesserungen vorgeschlagen.
Dieses Vorgehen hat das Ziel, Erfahrungen und etablierte Grundlagen in \gls{OeVGK18} im Kapitel \ref{Spezifikation OeVGK18} einfliessen zu lassen, um so eine allgemeine Akzeptanz erreichen zu können.

\subsubsection{Berechnungsmethodik ARE}
\label{Lösungsansätze:Berechnungsmethodik ARE}
Das Bundesamt für Raumentwicklung (ARE) entwickelt in ihrem Grundlagenbericht für die Beurteilung der Agglomerationsprogramme Verkehr und Siedlung~\cite{berechnung_are} eine Berechnungsmethodik der \acs{ÖV}-Güteklassen basierend auf der bestehende Norm (siehe Kapitel \ref{Aktuelle Situation:Schweizer Norm 640 290}).
Die entwickelte Berechnungsmethodik basiert auf den Fahrplandaten von HAFAS~\cite{sbb_hafas_spec}.
Dies hatte die Konsequenz, dass die Berechnungsmethodik wie sie in der Norm festgelegt ist, angepasst werden musste.
Im Folgenden sind die Anpassungen aufgeführt.

\paragraph{Bedienungsqualität der Haltestelle}~\\
\label{Berechnungsmethodik ARE:Bedienungsqualität der Haltestelle}
Die Verkehrsmittel lassen sich aus dem elektronischen Fahrplan~\cite{sbb_hafas_spec} nicht detailliert unterscheiden, wie es die Norm verlangt.
Dies hat die Konsequenz, dass nur folgenden Verkehrsmittel-Einteilung verwendet wird:

\begin{itemize}[noitemsep]
    \item Verkehrsmittelgruppe A
    \begin{itemize}
        \item Bahnknoten (mehrere Bahnlinien in verschiedenen Richtungen)
        \item Bahnlinien
    \end{itemize}
    \item Verkehrsmittelgruppe B
    \begin{itemize}
        \item Tram, Busse, Postautos, Rufbusse oder Schiffe
    \end{itemize}
    \item Verkehrsmittelgruppe C
    \begin{itemize}
        \item Seilbahnen
    \end{itemize}
\end{itemize}

Die Hauptlastrichtung kann aus den Fahrplandaten nicht extrahiert werden.
Somit werden alle Abfahrten zwischen 6.00 und 20.00 Uhr gezählt und anschliessend halbiert. Bei Endhaltestellen und Linien, welche nur in eine Richtung verlaufen, erfolgen Korrekturen.
Die genaue Art der Korrektur ist nicht weiter aufgeführt.
Als Stichtag für die Auswertung wird ein Werktag ausserhalb der Ferienzeit und der touristischen Hochsaison definiert.
Ebenfalls wir in reinen Arbeitsplatzgebieten mangels einheitlicher Definition keine Anpassung vorgenommen.
Verdichtungen in den Hauptverkehrszeiten sind in den gezählten Abfahrten inbegriffen.

\paragraph{Erreichbarkeit der Haltestelle}~\\
\label{Berechnungsmethodik ARE:Erreichbarkeit der Haltestelle}
Der Topografie wird keine Rechnung getragen, um eine landesweite Vergleichbarkeit sicherzustellen.

Die \acs{ÖV}-Güteklassen sind folgend zu interpretieren:

\begin{itemize}[noitemsep]
    \item Güteklasse A: Sehr gute Erschliessung
    \item Güteklasse B: Gute Erschliessung
    \item Güteklasse B: Mittelmässige Erschliessung
    \item Güteklasse D: Geringe Erschliessung
    \item Keine Güteklasse: Marginale oder keine \acs{ÖV}-Erschliessung
\end{itemize}

\subsubsection{Berechnungsmethodik Kanton Graubünden}
\label{Lösungsansätze:Berechnungsmethodik Kanton Graubünden}
Im technischen Bericht ''Definition \acs{ÖV}-Struktur / Erhebung  \acs{ÖV}-Güteklassen Kanton Graubünden''~\cite{oev-guteklasse-gr} werden Anpassungen der in der Norm beschriebenen Berechnungsmethodik der \acs{ÖV}-Güteklassen an die speziellen Bedürfnisse des Kanton Graubündens aber auch von allgemeiner Natur vorgenommen.
Anpassungen, welche spezielle Gegebenheiten des Kanton Graubündens betreffen, werden in der folgenden Zusammenstellung ausgeklammert, da sie sich für eine automatische Berechnung und für eine nationale Betrachtung nicht eignen.

\paragraph{Bedienungsqualität der Haltestelle}~\\
\label{Berechnungsmethodik Kanton Graubünden:Bedienungsqualität der Haltestelle}
Es werden zwei zusätzliche Haltestellenkategorieren (VI und VII) ergänzt und eine feinere Gliederung der Taktfrequenz eingeführt.

Auf die Unterscheidung zwischen ''Tram, Trolleybus, Autobus, Regionalbus'' und ''Ortsbus, lokaler Kleinbus'' wird verzichtet und in der ersteren Kategorie zusammengefasst.
Ebenfalls werden Seilbahnen mit Erschliessungsfunktion wie Bushaltestellen behandelt.

\begin{longtable}[c]{l p{4.0cm} p{4.0cm} p{4.0cm}}
        \midrule
        \textbf{Kursintervall}
                                & \textbf{Bahnknoten}
                                & \textbf{Bahnlinie}
                                & \textbf{Tram/Bus}\\
        \textit{< 5 min}
                                & I
                                & I
                                & II\\
        \cellcolor{red!25}\textit{5 - 10 min}
                                & I
                                & II
                                & III\\
        \cellcolor{red!25}\textit{11 - 19 min}
                                & II
                                & III
                                & IV\\
        \textit{20 - 39 min}
                                & III
                                & IV
                                & V\\
        \textit{40 - 60 min}
                                & IV
                                & V
                                & \cellcolor{red!25}VI\\
        \cellcolor{red!25}\textit{> 60 min}
                                &
                                & \cellcolor{red!25}VII
                                & \cellcolor{red!25}VII\\
        \bottomrule
    \caption{Ermittlung der Haltestellenkategorie Kanton Graubünden}
    \label{table:Ermittlung der Haltestellenkategorie Kanton Graubünden}
\end{longtable}

Somit werden nun Bushaltestellen, die halbstündlich und stündlich bedient werden, nicht mehr gleich kategorisiert.
Auch werden Haltestellen erfasst, welche seltener als stündlich einen Anschluss haben.
Durch die feinere Gliederung der Taktfrequenz erhält man eine bessere Güteklasse bei einem 10-Minuten-Takt im Vergleich mit einem 15-Minuten-Takt.

\paragraph{Erreichbarkeit der Haltestelle}~\\
\label{Berechnungsmethodik Kanton Graubünden:Erreichbarkeit der Haltestelle}
Die zusätzlichen Haltestellenkategorien (VI und VII) hat zur Folge, dass auch zusätzliche Güteklassen (E und F) eingeführt werden.

\begin{longtable}[c]{l p{3.3cm} p{3.3cm} p{3.3cm} p{3.3cm}}
        \midrule
        \textbf{}
                                & \textbf{< 300 m}
                                & \textbf{300 - 500 m}
                                & \textbf{501 - 750 m}
                                & \textbf{751 - 1000m}\\
        \textbf{I}
                                & Klasse A
                                & Klasse A
                                & Klasse B
                                & Klasse C\\
        \textbf{II}
                                & Klasse A
                                & Klasse B
                                & Klasse C
                                & Klasse D\\
        \textbf{III}
                                & Klasse B
                                & Klasse C
                                & Klasse D
                                &\\
        \textbf{IV}
                                & Klasse C
                                & Klasse D
                                &
                                &\\
        \textbf{V}
                                & Klasse D
                                &
                                &
                                &\\
        \cellcolor{red!25}\textbf{VI}
                                & \cellcolor{red!25}Klasse E
                                &
                                &
                                &\\
        \cellcolor{red!25}\textbf{VII}
                                & \cellcolor{red!25}Klasse F
                                &
                                &
                                &\\                                
        \bottomrule
    \caption{Ermittlung der Erreichbarkeit der Haltestelle Kanton Graubünden}
    \label{table:Ermittlung Erreichbarkeit der Haltestelle Kanton Graubünden}
\end{longtable}

Bushaltestellen haben eine maximale Erschliessungswirkung von 500 Meter, da grössere Distanzen von Passagieren als zu lange empfunden wird.
Somit werden nur Bahnknoten und Bahnlinien in den Kategorien > 500 Meter klassifiziert.
Die Topografie wird manuell berücksichtigt.

Hauptverkehrszeiten werden nicht gesondert gehandhabt.
Es zählt grundsätzlich das Angebot im gegebenen Zeitbereich.

Linien, welche grundsätzlich stündlich fahren, jedoch Taktlücken aufweisen um zu einen gewissen Zeitpunkt ein Angebot erfüllen zu können, werden nur in der Stundentakt-Gruppe klassifiziert, wenn sie nicht mehr als 2 Taktlücken aufweisen.

Es existieren Linienüberlagerungen mit einer ungleichen Verteilung.
So können zwei Linien stündlich verkehren und eine Haltestelle kurz nacheinander bedienen, welche dann als Halbstundentakt klassifiziert werden.
Der Passagier nimmt das Angebot aber als Stundentakt wahr.
Jedoch ist nicht weiter ersichtlich, wie das Problem behoben wird.

Die Bestimmung der Hauptlastrichtung erfolgt analog zur Berechnungsmethodik ARE.

% Brauchen uns Hinketakte zu interessieren und müssen wir das auflisten?

\subsubsection{Berechnungsmethodik Kanton Zürich}
\label{Lösungsansätze:Berechnungsmethodik Kanton Zürich}
Im Infoblatt ''ÖV-Güteklassen''~\cite{oev-guteklassen-zh} wird beschrieben, wie der Kanton Zürich die kantonalen ÖV-Güteklassen mit den Fahrplandaten des Zürcher Verkehrsverbundes (ZVV), angelehnt an die Berechnungsmethodik \acs{ARE}, berechnet.

\paragraph{Bedienungsqualität der Haltestelle}~\\
\label{Berechnungsmethodik Kanton Zürich:Bedienungsqualität der Haltestelle}
Es werden nur Haltestellen berücksichtigt, die von Bahn, Tram und/oder Bus bedient werden.
Seilbahnen und Schiffe werden explizit ausgeklammert.
Somit ergibt sich folgende Kategorisierung:
\begin{itemize}
    \itemsep -1.5em
    \item Bahnknoten (Bahnstationen mit S-Bahnlinien in min. 6 Richtungen und/oder IR-Anschluss)
    \item Bahnlinien
    \item Tram
    \item Bus
\end{itemize}

Der Kursintervall wird auf 05:30 bis 22:30 Uhr festgelegt (unter der Woche).
Handelt es sich um Endhaltestellen oder Linien, welche nur in eine Richtung verkehren, werden alle Abfahrten gezählt.
Zusätzlich werden wie im Kanton Graubünden  Haltestellen mit einem Kursintervall von über 60 Minuten berücksichtigt.

\begin{longtable}[c]{l p{2.9cm} p{2.8cm} p{2.8cm} p{2.8cm}}
        \midrule
        \textbf{Kursintervall}
                                & Bahnknoten
                                & Bahnlinie
                                & Tram
                                & Bus\\
        \textit{< 5 min}
                                & I
                                & I
                                & II
                                & II\\
        \textit{5 - 9 min}
                                & I
                                & II
                                & III
                                & III\\
        \textit{10 - 19 min}
                                & II
                                & III
                                & IV
                                & IV\\
        \textit{20 - 39 min}
                                & III
                                & IV
                                &
                                & V\\
        \textit{40 - 60 min}
                                &
                                & V
                                &
                                & VI\\
        \cellcolor{red!25}\textit{> 60 min}
                                &
                                &
                                &
                                & \cellcolor{red!25}VII\\
        \bottomrule
    \caption{Ermittlung der Haltestellenkategorie Kanton Zürich}
    \label{table:Ermittlung der Haltestellenkategorie Kanton Zürich}
\end{longtable}


\paragraph{Erreichbarkeit der Haltestelle}~\\
\label{Berechnungsmethodik Kanton Zürich:Erreichbarkeit der Haltestelle}
Auch im Kanton Zürich werden durch den zusätzlich abgebildeten Kursintervall (> 60 Minuten) zwei neue Güteklassen E und F eingeführt.

\begin{longtable}[c]{l p{3.3cm} p{3.3cm} p{3.3cm} p{3.3cm}}
        \midrule
        \textbf{}
                                & \textbf{< 300 m}
                                & \textbf{300 - 500 m}
                                & \textbf{501 - 750 m}
                                & \textbf{751 - 1000m}\\
        \textbf{I}
                                & Klasse A
                                & Klasse A
                                & Klasse B
                                & Klasse C\\
        \textbf{II}
                                & Klasse A
                                & Klasse B
                                & Klasse C
                                & Klasse D\\
        \textbf{III}
                                & Klasse B
                                & Klasse C
                                & Klasse D
                                &\\
        \textbf{IV}
                                & Klasse C
                                & Klasse D
                                & \cellcolor{red!25}Klasse E
                                &\\
        \textbf{V}
                                & Klasse D
                                & \cellcolor{red!25}Klasse E
                                & \cellcolor{red!25}Klasse E
                                &\\
        \cellcolor{red!25}\textbf{VI}
                                & \cellcolor{red!25}Klasse E
                                & \cellcolor{red!25}Klasse E
                                &
                                &\\
        \cellcolor{red!25}\textbf{VII}
                                & \cellcolor{red!25}Klasse F
                                & \cellcolor{red!25}Klasse F
                                &
                                &\\                                
        \bottomrule
    \caption{Ermittlung der Erreichbarkeit der Haltestelle Kanton Zürich}
    \label{table:Ermittlung Erreichbarkeit der Haltestelle Kanton Zürich}
\end{longtable}

Der Topografie wird keine Rechnung getragen, es wird strikt auf die Luftlinie gesetzt.
Sind Haltestellen nur durch Tram und/oder Bus erschlossen, werden analog zum Kanton Graubünden nur Distanzen bis zu 500 Meter berücksichtigt.
Die Erschliessung durch Bahnknoten und Bahnlinien wird bis 750 Meter mit den Güteklassen B – E und bis 1000 Meter mit den Güteklassen C-D berücksichtigt.
Diese Entscheidungen basieren auf ÖV-Angebotsverordnungen.

\subsection{Verbesserungsmöglichkeiten}
\label{Stand der Technik:Verbesserungsmöglichkeiten und Zusammenhang zu bestehenden Lösungen}
%TODO Abgrenzung zu Kapitel Problemstellung & Vision

Folgend werden identifizierte Schwachstellen der bestehenden Berechnungsmethodiken (siehe Kapitel \ref{Stand der Technik:Lösungsansätze}) aufgelistet und mögliche Optimierungen präsentiert.
Dabei behält man das Ziel der automatischen Berechnung hinter dem geistigen Auge.
Ob die definierten Kriterien mit der vorhandenen Datenbasis umsetzbar sind und ob diese für eine schweizweite Umsetzung tauglich ist, ist Bestandteil dieser Analyse.
Basierend auf diesen Verbesserungsmöglichkeiten, der \nameref{Lösungsansätze:Berechnungsmethodik ARE} und den Learnings der beiden kantonalen Berechnungsmethodiken wird im Kapitel \ref{Spezifikation OeVGK18} eine neue Berechnungsmethodik vorgeschlagen.

\subsubsection{Ermittlung der Erreichbarkeit der Haltestelle}
\label{Verbesserungsmöglichkeiten:Ermittlung der Erreichbarkeit der Haltestelle}

\paragraph{Problem}~\\
In allen Lösung wird die Luftlinie als massgebende Bewertung der Erreichbarkeit einer Haltestelle genommen.
Dies berücksichtigt weder die zusätzlicher Distanz durch verwinkelte Wege noch die Topographie.
Der Kanton Graubünden behebt diese Problematik mit manuellen Anpassungen.
Manuelle Eingriffe eigenen sich jedoch für eine automatisierte Lösung nicht.

\paragraph{Lösung}~\\
Um der Topographie und der Wegführung gerecht zu werden, wird einerseits der konkrete Weg entlang des Wege- und Strassennetzes für die Distanzberechnung berücksichtigt.
Auf der anderen Seiten wird die Topographie durch die Berechnung von \gls{Leistungskilometer}n berücksichtigt.

\subsubsection{Kursintervallberechnung}
\label{Verbesserungsmöglichkeiten:Kursintervallberechnung}
% TODO gefällt mir noch nicht so, zu wenig detailliert

\paragraph{Problem}~\\
Die Methodiken verwenden grundsätzlich einen Werktag ausserhalb der Ferienzeit und der touristischen Hochsaison und einen Bereich, welcher den Pendlerverkehr zu den Randzeit berücksichtigt, um den Kursintervall berechnen zu können.
Der Kanton Zürich geht noch einen Schritt weiter und vergrössert ihren Einzugsbereich.
Die Kurse am Wochenende und in der Nacht werden grundsätzlich ausgeklammert.
Für Privatpersonen, aber auch für Planer kann dieser Zeitraum relevant sein, um sich für einen Wohnort entscheiden oder um ein Angebot ausbauen zu können.

\paragraph{Lösung}~\\
Die Berechnung der \acs{ÖV}-Güteklassen wird für drei Zeiträume durchgeführt.
Zusätzlich zum Standardbereich, welcher den Pendlerverkehr berücksichtigt, wird der Verkehr ausserhalb des Pendlerverkehrs in der Nacht und das Wochenende berücksichtigt.
Natürlich muss hier eine andere Gewichtung angewendet werden, da zu Randzeiten Stundentakte weit häufiger vorkommen.

\subsubsection{Transportmittelkategorie}
\label{Verbesserungsmöglichkeiten:Transportmittelkategorie}

\paragraph{Problem}~\\
Die öffentlichen Verkehrsmittel werden in drei Verkehrsmittelgruppen gruppiert (siehe Kapitel \ref{Berechnungsmethodik ARE:Bedienungsqualität der Haltestelle}).
Wie in Kapitel \ref{subsystem:GTFS} beschrieben werden die Fahrplandaten im \acs{GTFS}-Format~\cite{gtfs_spec} gehalten. 
Es werden dabei 8 Verkehrsmittel-Typen definiert.
In Tabelle \ref{table:Mapping Verkehrsmittel-Typ Transportmittelkategorie} ist ein Mapping der definierten Verkehrsmittel-Typen und der Verkehrsmittelgruppen ersichtlich.

\begin{longtable}[ht]{l l}
        \midrule
        \textbf{GTFS Route Type} 
                                & \textbf{Verkehrsmittelgruppen}\\
        0 (Tram, Streetcar, Light rail)
                                & B\\
        1 (Subway, Metro)
                                & B\\
        2 (Rail)
                                & A\\
        3 (Bus)
                                & B\\
        4 (Ferry)
                                & B\\
        5 (Cable car)
                                & (C)\\
        6 (Gondola, Suspended cable car)
                                & C\\
        7 (Funicular)
                                & C\\            
        \bottomrule
    \caption{Mapping Verkehrsmittel-Typ Transportmittelkategorie}
    \label{table:Mapping Verkehrsmittel-Typ Transportmittelkategorie}
\end{longtable}

Stand 13.03.2018 kommt die GTFS Route Type Kategorie 5 gemäss den Fahrplandaten~\cite{geops_fahrplandaten} in der Schweiz nicht vor und kann vernachlässigt werden.

Schiffe und Seilbahnen haben nicht zwingend einen Erschliessungsfunktion und werden nur für touristische Zwecke benutzt, fliessen momentan aber ungefiltert in die Bewertung.

% TODO Tabelle steht momentan ein bisschen unnötigt da

\paragraph{Lösung}~\\
Bei der Berücksichtigung von Schiffen und Seilbahnen soll geprüft werden, ob diese eine Erschliessungsfunktion hat.

\subsubsection{Bestimmung der Bahnknoten}
\label{Verbesserungsmöglichkeiten:Bestimmung der Bahnknoten}

\paragraph{Problem}~\\
Bei der Unterscheidung von Bahnknoten und Bahnlinien ist bei der Berechnungsmethodik von ARE lediglich spezifiziert, dass ein Bahnknoten Bahnlinien in mehrere Richtungen hat (siehe Kapitel \ref{Lösungsansätze:Berechnungsmethodik ARE}).
In den ÖV-Güteklassen des Kanton Zürich werden Bahnknoten als "`Bahnstationen in mindestens 6 Richtungen und/oder IR-Anschluss"' (Kapitel \ref{Lösungsansätze:Berechnungsmethodik Kanton Zürich}) definiert.
Bei beiden Methodiken ist nicht klar, was unter einer Richtung genau zu verstehen ist.

\paragraph{Lösung}~\\
Für die Bestimmung von Bahnknoten werden von einer gegebenen Bahnstation aus diejenigen anderen Bahnstationen gezählt, die mit einem beliebigen Zug ohne Zwischenhalt erreicht werden können.
Konkret wird also für jeden (im definierten Zeitraum) an der Bahnstation haltenden Zug ermittelt, zu welcher Station er als nächstes fahren wird.
Die Anzahl Richtungen wird dann definiert als die Anzahl von verschiedenen Bahnstationen, die dadurch gezählt werden.
