
\section{Stand der Technik}
\label{Stand der Technik}

\subsection{Aktuelle Situation}
\label{Stand der Technik:Aktuelle Situation}

\subsubsection{Schweizer Norm 640 290}
\label{Aktuelle Situation:Schweizer Norm 640 290}

\acs{ÖV}-Güteklassen wurden erstmals mit der \ac{SN} 640 290~\cite{sn640290} des Vereins Schweizerischer Strassenfachleute (VSS) im Jahre 1993 für die Schweiz definiert.
Diese Norm enthält Richtwerte für die Bestimmung des Grenzbedarfs an Parkfeldern für Personenwagen und führte hierfür die Definition der \acs{ÖV}-Güteklassen ein.
Im Jahre 2006 wurde diese durch die \acs{SN} 640 281 abgelöst, welche nicht mehr auf die \acs{ÖV}-Güteklassen eingeht.
Es wurde seither keine allgemeingültige Definition verabschiedet.

Im Folgenden ist aufgeschlüsselt, was genau die \acs{SN} 640 290~\cite{sn640290} in Bezug auf \acs{ÖV}-Güteklassen definiert und wo Schwachstellen liegen.

\paragraph{Definition ÖV-Güteklassen}~\\
\label{Schweizer Norm 640 290:Definition ÖV-Güteklassen}
Grundsätzlich definieren \acs{ÖV}-Güteklassen die Qualität der Erschliessung durch den öffentlichen Verkehr.
Diese Qualität wird durch die \nameref{Definition ÖV-Güteklassen:Bedienungsqualität der Haltestelle} und die \nameref{Definition ÖV-Güteklassen:Erreichbarkeit der Haltestelle} festgesetzt.
Beide Kriterien sind folgend definiert.
Anmerken muss man, dass zu besserer Verständlichkeit und Übersichtlichkeit unter anderem zusätzliche Terminologie wie Transportmittelkategorie eingeführt wird.

\subparagraph{Bedienungsqualität der Haltestelle}~\\
\label{Definition ÖV-Güteklassen:Bedienungsqualität der Haltestelle}
Die öffentlichen Verkehrsmittel werden zuerst in die in Tabelle \ref{table:Transportmittelkategorien} ersichtlichen Transportmittelkategorien eingeteilt.

\begin{longtable}{l p{10.6cm}}
        \midrule
        \textbf{Transportmittelkategorie 1}
                                    & Bahnknoten: mehrere Bahnlinien in unterschiedliche Richtungen\\
        \textbf{Transportmittelkategorie 2}
                                    & Bahnlinie\\
        \textbf{Transportmittelkategorie 3}
                                    & Tramway, Trolleybuy, Autobus, Regionalbus, Ortsbus mit gutem Anschluss an Bahnlinie\\
        \textbf{Transportmittelkategorie 4}
                                    & Ortsbus, lokaler Kleinbus\\
        \bottomrule
    \caption{Transportmittelkategorien}
    \label{table:Transportmittelkategorien}
\end{longtable}

Die Transportmittel werden dabei in die Gruppe A und B zusammengefasst, welche für die Berechnung des durchschnittlichen Kursintervalls berücksichtigt werden.
Dabei werden für die Intervallberechnung alle Verkehrsmittel der Transportmittelkategorie in einer Gruppe berücksichtigt.

\begin{longtable}[c]{l | p{2.3cm} p{2.3cm} | p{2.3cm} p{2.3cm}}
        \midrule
        \textbf{}
                                & \multicolumn{2}{l|}{\textbf{Gruppe A}}
                                & \multicolumn{2}{l}{\textbf{Gruppe B}}\\
        \textbf{Transportmittelkategorie}
                                & \textit{1}
                                & \textit{2}
                                & \textit{3}
                                & \textit{4}\\
        \textbf{Kursintervall}
                                &
                                &
                                &
                                &\\
        \textit{< 5 min}
                                & I
                                & I
                                & II
                                & III\\
        \textit{5 - 9 min}
                                & I
                                & II
                                & III
                                & IV\\
        \textit{10 - 19 min}
                                & II
                                & III
                                & IV
                                & V\\
        \textit{20 - 39 min}
                                & III
                                & IV
                                & V
                                & V\\
        \textit{40 - 60 min}
                                & IV
                                & V
                                & V
                                & -\\
        \bottomrule
    \caption{Ermittlung der Haltestellenkategorie}
    \label{table:Ermittlung der Haltestellenkategorie}
\end{longtable}

Der Kursintervall ist der durchschnittliche Abstand zwischen Ankunft beziehungsweise Abfahrt aller Verkehrsmittel in der gleichen Gruppe zwischen 06:00 - 20:00 Uhr jeweils von Montag bis Freitag.
Die Norm macht hier Ausnahmen bei Verdichtungen in Hauptverkehrszeiten, Linienüberlagerungen und reinen Arbeitsplatzgebieten mit stark verdichtetem Fahrplan während Pendlerzeiten.

\subparagraph{Erreichbarkeit der Haltestelle}~\\
\label{Definition ÖV-Güteklassen:Erreichbarkeit der Haltestelle}
Basierend auf den vorhin definierten Haltestellenkategorien (I bis V) kann die Erreichbarkeit pro Haltestellenkategorien definiert werden.
Die Distanzen sind Luftlinien, wobei ein mittlerer Umwegfaktor von 20 bis 30\% berücksichtigt wird.

\begin{longtable}[c]{l p{3.3cm} p{3.3cm} p{3.3cm} p{3.3cm}}
        \midrule
        \textbf{}
                                & \textbf{< 300 m}
                                & \textbf{300 - 500 m}
                                & \textbf{501 - 750 m}
                                & \textbf{751 - 1000m}\\
        \textbf{I}
                                & Klasse A
                                & Klasse A
                                & Klasse B
                                & Klasse C\\
        \textbf{II}
                                & Klasse A
                                & Klasse B
                                & Klasse C
                                & Klasse D\\
        \textbf{III}
                                & Klasse B
                                & Klasse C
                                & Klasse D
                                & -\\
        \textbf{IV}
                                & Klasse C
                                & Klasse D
                                & -
                                & -\\
        \textbf{V}
                                & Klasse D
                                & -
                                & -
                                & -\\
        \bottomrule
    \caption{Ermittlung der Erreichbarkeit der Haltestelle}
    \label{table:Ermittlung Erreichbarkeit der Haltestelle}
\end{longtable}

Der Topographie wird Rechnung getragen, indem man bei schwierigen Gegebenheiten die nächste Klasse wählt oder die Distanz vergrössert.

\paragraph{Problematik}~\\
\label{Schweizer Norm 640 290:Problematik}
%TODO Abgrenzung zu Kapitel Problemstellung & Vision
Folgend werden identifizierte Schwachstellen der bestehenden Norm aufgelistet und mögliche Optimierungen präsentiert.
Dabei behält man das Ziel der automatischen Berechnung hinter dem geistigen Auge.
Ob die definierten Kriterien mit der vorhandenen Datenbasis umsetzbar sind und ob diese für eine schweizweite Umsetzung tauglich ist, ist Bestandteil dieser Analyse.
Bei den Optimierungen lehnen wir uns ebenfalls stark an die Erkenntnisse von~\cite{oev-guteklasse-gr}, welche die Norm an die spezifische Bedürfnisse des Kantons Graubünden anpasst hat und ~\cite{berechnung_are}, welche die Berechnungsmethodik basierend auf dieser Norm mit eigenen Anpassungen umgesetzt hat.

\subparagraph{Transportmittelkategorie}~\\
Die öffentlichen Verkehrsmittel werden in vier Kategorien gruppiert (siehe Tabelle \ref{table:Transportmittelkategorien}).
Wie in Kapitel \ref{subsystem:GTFS} beschrieben werden die Fahrplandaten im \acs{GTFS}-Format~\cite{gtfs_spec} gehalten. 
Es werden dabei 8 Verkehrsmittel-Typen definiert.
In Tabelle \ref{table:Mapping Verkehrsmittel-Typ Transportmittelkategorie} ist ein Mapping der definierten Verkehrsmittel-Typen und der Transportmittelkategorien der Norm ersichtlich.

\begin{longtable}[ht]{l l}
        \midrule
            \textbf{GTFS Route Type} 
            & \textbf{Transportmittelkategorie}\\
            
            0 (Tram, Streetcar, Light rail)
            & 3\\
            
            1 (Subway, Metro)
            & (3)\\
            
            2 (Rail)
            & 2\\
            
            3 (Bus)
            & 3\\
            
            4 (Ferry)
            & -\\
            
            5 (Cable car)
            & (3)\\
            
            6 (Gondola, Suspended cable car)
            & (3)\\
            
            7 (Funicular)
            & (3)\\            
        \bottomrule
    \caption{Mapping Verkehrsmittel-Typ Transportmittelkategorie}
    \label{table:Mapping Verkehrsmittel-Typ Transportmittelkategorie}
\end{longtable}

Die Norm unterscheidet unterschiedliche Bus-Typen (Autobus, Ortsbus, lokaler Kleinbus, etc.), welche ebenfalls für die Kategorisierung ausschlaggebend sind.
Die Fahrplandaten lassen so eine Unterscheidung auf Anhieb nicht zu.
Eine aufwändigere Berechnung wäre von Nöten, welche aber auch nur eine Annäherung an die in der Norm geforderte Unterscheidung wäre.

Fähren und Schiffen werden in der Norm ausgeklammert.
Eine Seeüberquerung per Fähre beispielsweise kann aber durchwegs ein Qualitätsmerkmal sein.

Die Norm gliedert Bahnen mit ''Tramcharakter'' der Transportkategorie 3 zu, was einen gewissen Raum zur Interpretation offen lässt.
Es lässt sich darüber streiten, ob eine Gondel mit einem Tram gleichgesetzt werden kann.
Stand 13.03.2018 kommt die GTFS Route Type Kategorie 5 gemäss den Fahrplandaten~\cite{geops_fahrplandaten} in der Schweiz nicht vor und kann vernachlässigt werden.

Die Fahrplandaten bieten eine feingranularere Unterscheidung, welche für die Kategorisierung genutzt werden kann.

\subparagraph{Ermittlung der Haltestellenkategorie}~\\
Wie in Tabelle \ref{table:Ermittlung der Haltestellenkategorie} ersichtlich ist, werden Verkehrsmittel in der Transportmittelkategorie 3 bei einem halb-stündlichen und stündlichen Takt in die gleiche Haltestellenkategorie eingeteilt.
Für jeden Nutzer des öffentlichen Verkehrs ist es von Bedeutung, ob er im schlimmsten Fall doppelt solange auf eine nächste Verbindung warten muss.
Haltestellen in weniger urbanen Gebieten, welche nur zu Pendlerzeiten bedient werden, finden in der aktuellen Definition keinen Platz.

\subparagraph{Ermittlung der Erreichbarkeit der Haltestelle}~\\
Es ist nicht genau definiert, wie die Topographie zu bewerten ist.
%TODO in GR wird gesagt, dass ein Zugangsweg > 500m als zu lange empfunden wird, auf welcher Basis? Wollen wir das beschreiben?

\subparagraph{Ermittlung des Kursintervalls und Hauptlastrichtung}~\\
Die geforderte Berechnung des Kursintervalls verlangt die Bestimmung der Hauptlastrichtung einer Linie.
Dies lässt sich basierend auf den Fahrplandaten nicht automatisch berechnen.
Ebenfalls fehlt eine allgemeingültige Definition, was unter einem Arbeitsplatzgebiet zu verstehen ist.
%TODO was genau ist der Nutzer den Einteilung in Gruppe A und B

\subparagraph{Raum für Interpretationen}~\\
Die Norm lässt für einige Situationen Handlungsspielraum offen.
Dies kann positiv wie auch negativ ausgelegt werden.
Betrachtet man es unter dem Gesichtspunkt, dass die \acs{ÖV}-Güteklassen-Definition für einen automatisierte Berechnung ohne manuelle Fremdeinwirkung und grossen Konfigurationsbedarf verwendet werden soll, ist dieser Punkt negativ zu gewichten.
Es ist unter anderem definiert, dass ein bei den Distanzen ein Umwegfaktor vom 20 bis 30\% zu berücksichtigen ist und je nach Topografie eine tiefere Klasse oder höhere Distanz gewählt wird.
Diese Definition verlangt, dass von Hand situativ entschieden wird oder ein Modell entwickelt wird, welche diese Entscheidung übernimmt und gibt aber keine Anhaltspunkte, in welchen Situationen wie stark angepasst werden muss.

\subparagraph{Unterschiedliche Berechnungsmethodiken}~\\
Es existieren unterschiedliche Berechnungsmethodiken, welche auf der Norm~\cite{sn640290} aufsetzen und diese modifizieren.
Eine nicht abschliessende Liste beinhaltet~\cite{berechnung_are},~\cite{oev-guteklasse-gr} und~\cite{oev-guteklassen-zh}.
Dies führt zum Schluss, dass mit der aktuellen Definition der \acs{ÖV}-Güteklassen keine allgemeingültige und akzeptierte Definition geschaffen wurde, welche schweizweit einsetzbar ist und regionale Einflüsse berücksichtigt.
Dies unterstützt die Schlussfolgerung, dass hier Handlungsbedarf existiert.

\subsection{Lösungsansätze}
\label{Stand der Technik:Lösungsansätze}

%TODO

\subsubsection{Bewertung}~\\

%TODO

\subsection{Verbesserungsmöglichkeiten}
\label{Stand der Technik:Verbesserungsmöglichkeiten}

%TODO